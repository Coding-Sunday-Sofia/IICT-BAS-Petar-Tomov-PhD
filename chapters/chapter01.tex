\chapter{Обзор на методите за прогнозиране с машинно самообучение}

\section{Прогнозиране на времеви редове}

Времевите редове са последователност от измервания, която следва строг ред във времето. Измерванията най-често са на равни интервали, но не винаги това е възможно или рационално \cite{Shen-01}. Изчислението на прогноза, базираща се на миналите стойности във времевия ред е своеобразна екстраполация \cite{Khashei-03}. В началото на 70-те години на XX век \cite{Gooijer-01} се предлагат линейни модели за прогнозиране на времеви редове – авторегресионен (Autoregressive) и плъзгащи средни (Moving Averages) \cite{Tealab-01}. При авторегресионния модел се възприема, че прогнозната стойност е линейна комбинация от стойностите в миналото. При плъзгащите средни прогнозната стойност е функция на случайни смущения - смущения, които са повлияли на времевия ред. Между двата модела е възможна комбинация под формата на авторегресионна интегрирана плъзгаща средна (Auto-Regressive Integrated Moving Average) \cite{Khashei-01}. При определянето на параметрите за авторегресията и интегрираната плъзгаща средна успешно могат да се приложат еволюционни алгоритми \cite{Cortez-01}. Линейните модели се оказват неприложими за множество времеви редове от реалната практика \cite{Bontempi-01}. Изкуствените неверонни мрежи се оказват един от подходите да се премине от линейност \cite{Zhang-03} към нелинейност в моделите. Трудностите при изкуствените невронни мрежи са свързани с по-големия брой параметри, които са трудни за определяне, като степен за обучение (learning rate) при обратно разпространение на грешката или размер на скрития слой \cite{Tang-01}. В някои разработки се правят опити да се автоматизира определянето на параметрите \cite{Yan-01}. Липсата на систематизиран подход за изграждане на изкуствени невронни мрежи \cite{Qi-01} е пряко свързан с определянето на параметрите. Линейност и нелинейност могат да се съчетаят в хибридни реализации, базирани на плъзгащи средни и изкуствени невронни мрежи \cite{Zhang-01}. Недостатък при изкуствените неверонни мрежи е нуждата от по-голям брой тренировъчни примери \cite{Lachtermacher-01}. Освен броя на тренировъчните примери, изключително съществен параметър се явява размерът на входния слой, който определя прозореца от стойности в миналия периода от миналото \cite{Chen-01}. С добавяне на обратни връзки в изкуствените невронни мрежи се постига ефект на кратковременна памет. Ефектът на кратковременната памет в изкуствените невронни мрежи често води до подобряване на получените прогнози \cite{Cao-02}. Ускорение в процеса по обучението на изкуствени невронни мрежи може да се получи чрез използването на вероятностни невронни мрежи (Probabilistic Neural Networks) \cite{Khashei-02}. Комбинацията от линейни уравнения за постигането на разграничение при нелинейни класове се постига и чрез машина с поддържащи вектори (Support Vector Machines) \cite{Kyoung-jae-01}. Машините с поддържащи вектори се явяват подобрение на изкуствените невронни мрежи обучавани по правилото за обратно разпространение на грешката. Времето за обучение се намалява, докато достоверността на прогнозите се увеличава \cite{Tay-01}. Ефективността на машините с поддържащи вектори може да се подобри, когато се използват алгоритми за адаптиране на параметрите \cite{Cao-01}. При времеви редове с ясно изразен тренд и сезонност \cite{Zhang-04}, премахването им може да подобри възможностите за генериране на прогноза \cite{Zhang-02}, включително и в комбинация с изкуствени неверонни мрежи \cite{Jain-01}. От оригиналния времеви ред се изваждат стойностите на тренда и подбрани циклични функции \cite{Nelson-01}. При слабо изразена сезонност, по-удачно може да се окаже сезонността да не се премахва \cite{Hamzacebi-01}. Възможност за генериране на прогнози дават и алгоритмите за разпознаване на образи (Patterns Recognition). Времевият ред може да бъде изследван за наличето на конкретни шаблони/образи и прогнозата да се ознвава на появата на конкретен шаблон \cite{Singh-01}. При много зашумени времеви редове е удачно да се приложи предварителна обработка на данните, така че шумът да се премахне или редуцира \cite{Lu-01}.

