\addcontentsline{toc}{chapter}{Заключение – резюме на получените резултати}
\chapter*{Заключение – резюме на получените резултати}
\markboth{Заключение – резюме на получените резултати}{}

В почти всички области на човешката дейност се наблюдават времеви редове. Времевите редове намират различни приложения в икономическото прогнозиране, прогнозирането на продажби, бюджетен анализ, процеса на контрол по качеството, анализ на фондовите пазари и др. В тази връзка са разработени много начини за прогнозиране на финансови времеви редове, но като един от най-обещаващите, се откроява прогнозирането с изкуствени невронни мрежи. Характерно за изкуствените невронни мрежи е, че те са много ефективен инструмент, след като веднъж са обучени. Процесът на обучение, от своя страна, често отнема твърде дълго време и се нуждае от голямо количество изчислителни ресурси. Това налага търсеното на нови подходи, свързани с намаляване на времето за обучение на изкуствените невронни мрежи. Поради това, разработването на хибридни алгоритми, целящи ускоряване на обучението при изкуствени невронни мрежи от тип многослоен перцептрон, за целите на прогнозирането на времеви редове, е актуално научноизследователско направление.

Предложени са хибридни алгоритми, които комбинират точни числени алгоритми с евристични алгоритми за оптимизация на теглата в изкуствени невронни мрежи. Възможността за паралелна обработка при еволюционните алгоритми, от своя страна позволява обучението на изкуствените невронни мрежи да се организира в разпределена среда от мобилни устройства.

В процеса на обучение на ИНМ, при определяне на теглата е предложено да се използва генетичен алгоритъм. В него се прилагат операциите – селекция, кръстосване и мутация. Предложен и анализиран е нов оператор за селекция, основан на създаването на рекурсивни поколения при процедура за рекурсивно спускане. На всяко ниво на рекурсията всички индивиди от популацията се чифтосват помежду си. Това чифтосване се прави по алгоритъма на грубата сила. Този подход добре се комбинира с локално търсене. Кръстосването се прилага, докато има излъчен по-добър индивид в поколението. Така описания хибриден подход, съчетаващ алгоритмите локално търсене и грубата сила води до по-добри резултати за сметка на по-дългото време за изчисление. Разгледани са и възможностите за апроксимиране на криви към множество точки. За инкрементална апроксимация на времеви редове се използват уравнение на права и ред от синус функции. Предложен е подход за пресмятане на коефициентите на синус функциите с оптимизатор, базиран на еволюция на разликите и рояк от частици. Използваните алгоритми се прилагат на хибриден принцип. Построената крива се използва за генериране на прогноза, извън диапазона на известните измерени точки.

Предложена е алтернатива на производна за активационната функция в изкуствени невронни мрежи. Направено е сравнение между предложената алтернативна функция и първа производна на периодична затихваща активационна функция. Алтернативната функция дава по-добри резултати по отношение на бързодействие и допусната грешка. Предложен е и алгоритъм за обучение на изкуствени невронни мрежи от тип многослоен перцептрон в разпределена среда. Като средство за оптимизация на бавно изчислими целеви функции е предложен генетичен алгоритъм. Той лесно се подлага на паралелна обработка. За един и същи период от време могат да бъдат получени множество решения на бавно изчислимите целеви функции.

Направена е програмна реализация за хибридно използване на градиентни числени и евристични алгоритми за оптимизация на теглата в изкуствени невронни мрежи. Реализираната система е базирана на мобилни разпределени изчисления за прогнозиране на финансови времеви редове. Разработено е мобилно приложение за Android операционна система. То дава възможност на потребителя да даде своя вот, затова как очаква да се промени курса за конкретна валутна двойка, за визуализация на процеса по обучение и представя прогнозната стойност от обучената на устройството невронна мрежа.

Направен е сравнителен анализ на точни числени и евристични алгоритми. Като входни данни са използвани базова форма на времеви ред, следващ синус функция и сложна форма на времеви ред от цената на дигиталната валута биткойн. От проведените експерименти и получените резултати е установено, че при точните числени алгоритми с проста структура на времеви ред, обучението с обратно разпространение на грешката има най-добра ефективност. При сложна входна структура най-добра ефективност дава еластичното (resilient) обучение. При евристичните алгоритми с простра форма на времеви ред най-добра ефективност дава генетичният алгоритъм. При сложна структура добра ефективност показва алгоритъма еволюционна стратегия.

Получените резултати, описани в настоящия дисертационен труд, могат да се обобщят в следните научни и научно-приложни приноси/резултати:

\begin{description}
\item 1. Направен е обзорен анализ и класификация на алгоритмите за обучение на изкуствени невронни мрежи от тип многослоен перцептрон. Разгледани са най-използваните начини за прогнозиране на времеви редове и по какъв начин машинното самообучение се прилага в тази проблемна област. Установено е, че алгоритъмът с обратно разпространение на грешката, който е от групата на точните числени, градиентни алгоритми, е най-често използвания за обучение на изкуствени невронни мрежи. Установено е също така, че този алгоритъмът много добре се допълва с евристичните, еволюционни алгоритми за глобална оптимизация, които от своя страна се поддават на изключително висока степен за паралелна обработка;
\item 2. За целите на процеса на обучение на ИНМ, при определяне на теглата, е предложено да се използва генетичен алгоритъм, който разчита на операциите като селекция, кръстосване и мутация. Поради това е предложен нов оператор за селекция, основан на създаването на поколения при процедура за рекурсивно спускане, което осигурява търсеното бързодействие на използваните евристични алгоритми;
\item 3. Инкременталната апроксимация на времевите редове най-често се реализира от уравнение на права и ред от синус функции. За постигане на по-добро апроксимиране, е предложен е подход за пресмятане на коефициентите на синус функциите с оптимизатор, базиран на еволюция на разликите и рояк от частици;
\item 4.	Предложена е алтернатива на производна за активационната функция в изкуствени невронни мрежи. Получените резултати показват по-добро бързодействие и допусната грешка в полза на предложената алтернативна функция отколкото при използването на първа производна на периодична затихваща активационна функция;
\item 5.	Предложен е генетичен алгоритъм за обучение на изкуствени невронни мрежи от тип многослоен перцептрон в разпределена среда, което прави възможно използването му при паралелна обработка;
\item 6.	Предложена е софтуерна архитектура, позволяваща реализиране на мобилни разпределени изчисления, базираща се на предложените хибридни алгоритми. Програмната реализация за хибридно използване на градиентни числени и евристични алгоритми за оптимизация на теглата в изкуствени невронни мрежи е реализирана в мобилно приложение за Android.
\end{description}

\section*{Насоки за бъдещи изследвания}

Постигнатото в операционната система Android би могло да бъде сходна цел за постигане в операционната система iOS, на компанията Apple. За разлика от Android, iOS е изцяло затворена операционна система. Моделът за разпространение на приложения под iOS също би довел до допълнителни затруднения. Друга насока за развитие на идеите от дисертационния труд е ориентирана към операционната система KaiOS. Тази операционна система все още няма голямо разпространение и е насочена предимно към по-маломощни мобилни устройства, но може да се окаже изключително продуктивна посока за допълнителни научни изследвания.

По отношение на използваните алгоритми и софтуерни библиотеки, съ-ществуват множество възможности за подобряване на наличните алгоритми и добавяне на нови такива. Използваните софтуерни библиотеки са с отворен код, което дава огромна свобода за изследване на програмния код, неговото дописване и оптимизиране, както и разширяването му с нови алгоритми.

Някои от областите с потенциално приложение на изкуствените невронни мрежи са управление на автономни системи, прогнозиране на работното натоварване на служители, прогнозиране на присъствието на служителите в офиса, при извършване на преброяване на населението и др. Към някои от тези потенциални приложения са  насочени и част от бъдещите изследвания.

\newpage

\section*{Публикации по темата на дисертационния труд}

\begin{itemize}
\item Mateeva, G., Tomov, P., Parvanov, D., Petrov, P., Kostadinov, G., Balabanov, T.: Some Capabilities of Android OS for Distributed Computing. Proceedings of Big Data, Knowledge and Control Systems Engineering BdKCSE’21, 2021, 1-6, ISBN 978-1-6654-1043-4.

\item Tomov, P.: Encog Gradient Training Algorithms Evaluation. Problems of Engineering Cybernetics and Robotics, vol. 77, 2021, 11-19, ISSN 2738-7356.

\item Tomov, P.: Multilayer Perceptron Fast Prototyping with Differential Evolution and Particle Swarm Optimization in LibreOffice Calc. Problems of Engineering Cybernetics and Robotics, vol. 75, 2021, 5-14, ISSN 2738-7356.

\item Tomov, P., Zankinski, I., Balabanov, T.: Training of Artificial Neural Networks for Financial Time Series Forecasting in Android Service and Widgets. Problems of Engineering Cybernetics and Robotics, no. 71, Institute of Information and Communication Technologies - Bulgarian Academy of Sciences, 2019, 50-56, ISSN 1314-409X.

\item Tomov, P., Zankinski, I., Balabanov, T.: Server Side Vote Clustering in Human-Computer Distributed Computing. Information Technologies and Control, no. 2, John Atanasoff Society of Automatics and Informatics, 2019, 15-19, ISSN 2367-5357.

\item Tomov, P., Zankinski, I., Barova, M.: Artificial Neural Networks Time Series Forecasting with Android Live Wallpaper Technology. Proceedings of the International Conference Numerical Methods for Scientific Computations and Advanced Applications NMSCAA’18, May 28-31, 2018, Hissarya, Fastumprint, 2018, 76-79, ISBN 978-954-91700-7-8.

\item Tomov, P., Zankinski, I., Barova, M.: Mobile Alternative of the Moneybee Project For Financial Forecasting. Proceedings of the Annual University Scientific  Conference of the National Military University Vasil Levski, June 14-15, 2018, Veliko Tarnovo, Innovetion and Sustainability Academy – ISA, 2018, 1085-1089, ISSN 2367-7481.

\item Zankinski, I., Barova, M., Tomov, P.: Hybrid Approach Based on Combination of Backpropagation and Evolutionary Algorithms for Artificial Neural Networks Training by Using Mobile Devices in Distributed Computing Environment. Proceedings of 11th International Conference on Large-Scale Scientific Computations LSSC'17, June 5-9, 2017, Sozopol, Bulgaria, 2017, 425-434, ISBN 978-3-319-73440-8.

\item Tomov, P., Monov, V.: Modeling and Analysis of Time Series. Proceedings of International Scientific Conference UniTech’17, Gabrovo, November 17-18, 2017, vol. II, University publishing house Vasil Aprilov, 2017, 404-409, ISSN 1313-230X.

\item Tomov, P., Monov, V.: Artificial Neural Networks and Differential Evolution Used for Time Series Forecasting in Distributed Environment. Proceedings of the International Conference Automatics and informatics, October 4-5, 2016, Sofia, Bulgaria, Federation of the scientific engineering unions, John Atanasoff Society of Automatics and Informatics, 2016, 129-132, ISSN 1313-1850.

\item Keremedchiev, D., Barova, M., Tomov, P.: Mobile Application as Distributed Computing System for Artificial Neural Networks Training Used in Perfect Information Games, International Scientific Conference UniTech’16, Gabrovo, University publishing house Vasil Aprilov, 2016, 389-393, ISSN 1313-230X.
\end{itemize}\newpage

\section*{Забелязани цитирания}

\begin{itemize}
\item Tomov, P., Zankinski, I., Balabanov, T.: Training of Artificial Neural Networks for Financial Time Series Forecasting in Android Service and Widgets. Problems of Engineering Cybernetics and Robotics, no. 71, Institute of Information and Communication Technologies - Bulgarian Academy of Sciences, 50--56, 2019. ISSN 1314-409X
	\begin{itemize}
	\item Borissova, D., Dimitrova, Z., Dimitrov, V.: How to Support Teams to be Remote and Productive: Group Decision-Making for Distance Collaboration Software Tools. Information \& Security: An International Journal, vol. 46, no. 1, 36--52, 2020. ISSN 0861-5160 DOI 10.11610/isij.4603

	\item Terzieva, M., Karastoyanov, D.: ICT for Innovation in Advanced Banking. Problems of Engineering Cybernetics and Robotics, vol. 73, 47--54, 2020. ISSN 2738-7356 DOI 10.7546/PECR.73.20.05

	\item Borissova, D., Dimitrova, Z. Garvanova, M., Garvanov, I., Cvetkova, P., Dimitrov, V., Pandulis, A.: Two-stage Decision-Making Approach to Survey the Excessive Usage of Smart Technologies, Problems of Engineering Cybernetics and Robotics, vol. 73, 3--16, 2020. ISSN 2738-7356 DOI 10.7546/PECR.73.20.01
	\end{itemize}

\item Zankinski, I., Barova, M., Tomov, P.: Hybrid Approach Based on Combination of Backpropagation and Evolutionary Algorithms for Artificial Neural Networks Training by Using Mobile Devices in Distributed Computing Environment. Proceedings of 11th International Conference on Large-Scale Scientific Computations LSSC'17, June 5-9, 2017, Sozopol, Bulgaria, 425--434, 2017. ISBN 978-3-319-73440-8 DOI 10.1007/978-3-319-73441-5\_46
	\begin{itemize}
	\item Koprinkova-Hristova, P.: Research on Artificial Neural Networks in Bulgarian Academy of Sciences. Research in Computer Science in the Bulgarian Academy of Sciences, vol. 934, 287--304, 2021. ISBN 978-3-030-72283-8 DOI 10.1007/978-3-030-72284-5\_14
	\end{itemize}

\item Tomov, P., Monov, V.: Artificial Neural Networks and Differential Evolution Used for Time Series Forecasting in Distributed Environment. Proceedings of the International Conference Automatics and informatics, October 4-5, 2016, Sofia, Bulgaria, Federation of the scientific engineering unions, John Atanasoff Society of Automatics and Informatics, 129--132, 2016. ISSN 1313-1850
	\begin{itemize}
	\item Balabanov, T.: Long Short Term Memory in MLP Pair. Proceedings of the International Scientific Conference UniTech, vol. II, 375--379, 2017. ISSN:1313-230X
	\item Balabanov, T., Atanasova, T., Blagoev, I.: Activation Function Permutation for Multilayer Perceptron Training. Proceedings of International Conference on Big Data, Knowledge and Control Systems Engineering, 9--14 , 2018. ISSN 2367-6450.
	\item Balabanov, T., Zankinski, I., Ketipov, R.: Weights Permutation in Multilayer Perceptron. Proceedings of International Conference on Big Data, Knowledge and Control Systems Engineering, 23--27 , 2018. ISSN 2367-6450
	\item Blagoev, I., Sevova, J., Kolev, K.: Dual MLP Pairs with Hidden Layer Sharing. Proceedings of 32nd International Conference on Information Technologies, 81--86, 2018. ISSN 1314-1023
	\item Blagoev, I., Sevova, J., Kolev, K.: Artificial Neural Network Activation Function Optimization with Genetic Algorithms. Proceedings of the International Conference Numerical Methods for Scientific Computations and Advanced Applications, 16--19, 2018. ISBN 978-954-91700-7-8
	\end{itemize}

\item Keremedchiev, D., Barova, M., Tomov, P.: Mobile Application as Distributed Computing System for Artificial Neural Networks Training Used in Perfect Information Games. International Scientific Conference UniTech’16, Gabrovo, University publishing house Vasil Aprilov, 389--393, 2016. ISSN 1313-230X
	\begin{itemize}
	\item Balabanov, T.: Long Short Term Memory in MLP Pair. Proceedings of the International Scientific Conference UniTech, vol. II, 375--379, 2017. ISSN:1313-230X
	\item Blagoev, I., Sevova, J., Kolev, K.: Dual MLP Pairs with Hidden Layer Sharing. Proceedings of 32nd International Conference on Information Technologies, 81--86, 2018. ISSN 1314-1023
	\item Blagoev, I., Sevova, J., Kolev, K.: Artificial Neural Network Activation Function Optimization with Genetic Algorithms. Proceedings of the International Conference Numerical Methods for Scientific Computations and Advanced Applications, 16--19, 2018. ISBN 978-954-91700-7-8
	\end{itemize}
\end{itemize}

\section*{Награди}

Награда в състезание за глобална скалируема оптимизация, провело се като част от Международната конференция за високопроизводителни изчисления, 2019 година.

\newpage
\section*{Декларация за оригиналност на резултатите}

\vspace{1cm}

Декларирам, че настоящата дисертация съдържа оригинални резултати, получени при проведени от мен научни изследвания, с подкрепата и съдействието на научния ми ръководител. Резултатите, които са получени, описани и/или публикувани от други учени, са надлежно и подробно цитирани в библиографията.

Настоящата дисертация не е прилагана за придобиване на научна степен в друго висше училище, университет или научен институт.

\vspace{2cm}

\begin{tabular}{ c c c c }
Дата: & .......................................... & Подпис: & .......................................... \\ 
& гр. София & & / Петър Томов / \\  
\end{tabular}
