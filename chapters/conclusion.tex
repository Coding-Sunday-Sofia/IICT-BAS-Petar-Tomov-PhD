\addcontentsline{toc}{chapter}{Заключение}
\chapter*{Заключение}
\markboth{Заключение}{}

Дисертационният труд разглежда проблеми от машинното самообучение, като акцентира върху алгоритмите за обучение на изкуствени невронни мрежи. Комбинирани са точни числени методи за обучение на изкуствени невронни мрежи с евристични алгоритми за оптимизация на теглата в изкуствени невронни мрежи. Залагайки на значителните възможности за паралелна обработка при еволюционните алгоритми, обучението на изкуствените невронни мрежи се организира в разпределена среда от мобилни устройства.

\section*{Резюме на получените резултати}

В дисертационния труд са постигнати резултати с научно-приложен и приложен характер, както следва:

\subsection*{Научно-приложни резултати}

\begin{description}
\item 1. Предложени са нови алгоритми за прогнозиране на времеви редове, чрез апроксимация със синус функции;
\item 2. Предложени са нови алгоритми за селекция при генетичните алгоритми, базирани на рекурсивно спускане, пълно изчерпване и локално търсене;
\item 3. Предложени са алтернативни активационни функции за изкуствени неврони;
\end{description}

\subsection*{Приложни резултати}

\begin{description}
\item 4. Разработен е програмен код за хубридно използване на градиентни числени методи за обучение на изкуствени невронни мрежи и евристични алгоритми за оптимизация на теглата в изкуствени невронни мрежи;
\item 5. Разработен е програмен код за визуализация на процеса по обучението на изкуствени невронни мрежи със средствата на операционната система Android;
\item 6. Разработен е програмен код за събиране на потребителски вот при реализация на човек-машина разпределени изчисления;
\end{description}

\section*{Насоки за бъдещи изследвания}

Основните усилия в дисертационния труд е разработката на алгоритми за обучение на изкуствени невронни мрежи в мобилна разпределена среда, базирана на операционната система Android. По настояще, това е най-широко използваната операционна система. Факт, който частично се дължи на части от операционната система, които са с отворен код. Постигнатото в операционната система Android би могло да бъде сходна цел за постигане в операционната система iOS, на компанията Apple. За разлика от Android, iOS е изцяло затворена операционна система. Моделът за разпространение на приложения под iOS също би довел допълнителни затруднения. Трета насока за развитие на идеите от дисертационния труд би била операционната система KaiOS. Тази операционна система все още няма голямо разпространение и е насочена предимно към по-маломощни мобилни устройства, но може да се окаже изключително продуктивна посока за допълнителни научни изследвания. 

По отношение на използваните алгоритми и софтуерни библиотеки, съществуват множество възможности за подобряване на наличните алгоритми и добавяне на нови такива. Използваните софтуерни библиотеки са с отворен код, което дава огромна свобода за изследване на програмния текст, дописване и оптимизиране на този програмен текст, както и разширяването му с нови алгоритми. 

\section*{Публикации по темата на дисертационния труд}

\begin{itemize}
\item Tomov, P., Zankinski, I., Balabanov, T.: Training of Artificial Neural Networks for Financial Time Series Forecasting in Android Service and Widgets. Problems of Engineering Cybernetics and Robotics, no. 71, Institute of Information and Communication Technologies - Bulgarian Academy of Sciences, 2019, 50-56, ISSN 1314-409X.

\item Tomov, P., Zankinski, I., Balabanov, T.: Server Side Vote Clustering in Human-Computer Distributed Computing. Information Technologies and Control, no. 2, John Atanasoff Society of Automatics and Informatics, 2019, 15-19, ISSN 2367-5357.

\item Tomov, P., Zankinski, I., Barova, M.: Artificial Neural Networks Time Series Forecasting with Android Live Wallpaper Technology. Proceedings of the International Conference Numerical Methods for Scientific Computations and Advanced Applications NMSCAA’18, May 28-31, 2018, Hissarya, Fastumprint, 2018, 76-79,  ISBN 978-954-91700-7-8.

\item Tomov, P., Zankinski, I., Barova, M.: Mobile Alternative of the Moneybee Project For Financial Forecasting. Proceedings of the Annual University Scientific  Conference of the National Military University Vasil Levski, June 14-15, 2018, Veliko Tarnovo, Innovetion and Sustainability Academy – ISA, 2018, 1085-1089, ISSN 2367-7481.

\item Zankinski, I., Barova, M., Tomov, P.: Hybrid Approach Based on Combination of Backpropagation and Evolutionary Algorithms for Artificial Neural Networks Training by Using Mobile Devices in Distributed Computing Environment. Proceedings of 11th International Conference on Large-Scale Scientific Computations LSSC'17, June 5-9, 2017, Sozopol, Bulgaria, 2017, 425-434, ISBN 978-3-319-73440-8.

\item Tomov, P., Monov, V.: Modeling and Analysis of Time Series. Proceedings of International Scientific Conference UniTech’17, Gabrovo, November 17-18, 2017, vol. II, University publishing house Vasil Aprilov, 2017, 404-409, ISSN 1313-230X.

\item Tomov, P., Monov, V.: Artificial Neural Networks and Differential Evolution Used for Time Series Forecasting in Distributed Environment. Proceedings of the International Conference Automatics and informatics, October 4-5, 2016, Sofia, Bulgaria, Federation of the scientific engineering unions, John Atanasoff Society of Automatics and Informatics, 2016, 129-132, ISSN 1313-1850.

\item Keremedchiev, D., Barova, M., Tomov, P.: Mobile Application as Distributed Computing System for Artificial Neural Networks Training Used in Perfect Information Games, International Scientific Conference UniTech’16, Gabrovo, University publishing house Vasil Aprilov, 2016, 389-393, ISSN 1313-230X.
\end{itemize}

\section*{Забелязани цитирания}

\begin{itemize}
\item Tomov, P., Zankinski, I., Balabanov, T.: Training of Artificial Neural Networks for Financial Time Series Forecasting in Android Service and Widgets. Problems of Engineering Cybernetics and Robotics, no. 71, Institute of Information and Communication Technologies - Bulgarian Academy of Sciences, 50--56, 2019. ISSN 1314-409X
	\begin{itemize}
	\item Borissova, D., Dimitrova, Z., Dimitrov, V.: How to Support Teams to be Remote and Productive: Group Decision-Making for Distance Collaboration Software Tools. Information \& Security: An International Journal, vol. 46, no. 1, 36--52, 2020. ISSN 0861-5160 DOI 10.11610/isij.4603

	\item Terzieva, M., Karastoyanov, D.: ICT for Innovation in Advanced Banking. Problems of Engineering Cybernetics and Robotics, vol. 73, 47--54, 2020. ISSN 2738-7356 DOI 10.7546/PECR.73.20.05

	\item Borissova, D., Dimitrova, Z. Garvanova, M., Garvanov, I., Cvetkova, P., Dimitrov, V., Pandulis, A.: Two-stage Decision-Making Approach to Survey the Excessive Usage of Smart Technologies, Problems of Engineering Cybernetics and Robotics, vol. 73, 3--16, 2020. ISSN 2738-7356 DOI 10.7546/PECR.73.20.01
	\end{itemize}

\item Zankinski, I., Barova, M., Tomov, P.: Hybrid Approach Based on Combination of Backpropagation and Evolutionary Algorithms for Artificial Neural Networks Training by Using Mobile Devices in Distributed Computing Environment. Proceedings of 11th International Conference on Large-Scale Scientific Computations LSSC'17, June 5-9, 2017, Sozopol, Bulgaria, 425--434, 2017. ISBN 978-3-319-73440-8 DOI 10.1007/978-3-319-73441-5\_46
	\begin{itemize}
	\item Koprinkova-Hristova, P.: Research on Artificial Neural Networks in Bulgarian Academy of Sciences. Research in Computer Science in the Bulgarian Academy of Sciences, vol. 934, 287--304, 2021. ISBN 978-3-030-72283-8 DOI 10.1007/978-3-030-72284-5\_14
	\end{itemize}

\item Tomov, P., Monov, V.: Artificial Neural Networks and Differential Evolution Used for Time Series Forecasting in Distributed Environment. Proceedings of the International Conference Automatics and informatics, October 4-5, 2016, Sofia, Bulgaria, Federation of the scientific engineering unions, John Atanasoff Society of Automatics and Informatics, 129--132, 2016. ISSN 1313-1850
	\begin{itemize}
	\item Balabanov, T.: Long Short Term Memory in MLP Pair. Proceedings of the International Scientific Conference UniTech, vol. II, 375--379, 2017. ISSN:1313-230X
	\item Balabanov, T., Atanasova, T., Blagoev, I.: Activation Function Permutation for Multilayer Perceptron Training. Proceedings of International Conference on Big Data, Knowledge and Control Systems Engineering, 9--14 , 2018. ISSN 2367-6450.
	\item Balabanov, T., Zankinski, I., Ketipov, R.: Weights Permutation in Multilayer Perceptron. Proceedings of International Conference on Big Data, Knowledge and Control Systems Engineering, 23--27 , 2018. ISSN 2367-6450
	\item Blagoev, I., Sevova, J., Kolev, K.: Dual MLP Pairs with Hidden Layer Sharing. Proceedings of 32nd International Conference on Information Technologies, 81--86, 2018. ISSN 1314-1023
	\item Blagoev, I., Sevova, J., Kolev, K.: Artificial Neural Network Activation Function Optimization with Genetic Algorithms. Proceedings of the International Conference Numerical Methods for Scientific Computations and Advanced Applications, 16--19, 2018. ISBN 978-954-91700-7-8
	\end{itemize}

\item Keremedchiev, D., Barova, M., Tomov, P.: Mobile Application as Distributed Computing System for Artificial Neural Networks Training Used in Perfect Information Games. International Scientific Conference UniTech’16, Gabrovo, University publishing house Vasil Aprilov, 389--393, 2016. ISSN 1313-230X
	\begin{itemize}
	\item Balabanov, T.: Long Short Term Memory in MLP Pair. Proceedings of the International Scientific Conference UniTech, vol. II, 375--379, 2017. ISSN:1313-230X
	\item Blagoev, I., Sevova, J., Kolev, K.: Dual MLP Pairs with Hidden Layer Sharing. Proceedings of 32nd International Conference on Information Technologies, 81--86, 2018. ISSN 1314-1023
	\item Blagoev, I., Sevova, J., Kolev, K.: Artificial Neural Network Activation Function Optimization with Genetic Algorithms. Proceedings of the International Conference Numerical Methods for Scientific Computations and Advanced Applications, 16--19, 2018. ISBN 978-954-91700-7-8
	\end{itemize}
\end{itemize}

\section*{Награди}

Награда в състезание за глобална скалируема оптимизация, провело се като част от Международната конференция за високопроизводителни изчисления, 2019 година.

\newpage
\section*{Декларация за оригиналност на резултатите}

\vspace{1cm}

Декларирам, че дисертацията съдържа оригинални резултати, получени при проведени от мен, научни изследвания с подкрепата и съдействието на научния ми ръководител.

Резултатите, които са получени, описани и/или публикувани от други учени, са коректно и подробно цитирани в библиографията.

Настоящият дисертационен труд не е прилаган за придобиване на научна степен в друго висше училище, университет или научен институт.

\vspace{2cm}

\begin{tabular}{ c c c c }
Дата: & .......................................... & Подпис: & .......................................... \\ 
& гр. София & & / Петър Томов / \\  
\end{tabular}
