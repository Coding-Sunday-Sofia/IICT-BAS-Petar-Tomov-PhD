\addcontentsline{toc}{chapter}{Увод}
\chapter*{Увод}
\markboth{УВОД}{}

Ежедневно човек се сблъсква с много величини, които се променят постоянно. Често пъти те са случайни, като зависят от различни условия. Могат да имат циклично променящ се характер и да проявяват тенденция. Факторите, които влияят върху една величина са много и не могат да бъдат оценени. Затова те се разглеждат като статистически величини. Анализът на такива случайни процеси, с помощта на статистиката, обикновено се извършва, след като те се дискретизират и се представят като серия от данни. Въвежда се понятието период, който е постоянна величина и се нарича време на дискретизация. Дискретизираните случайни величини, които са подредени в хронологичен ред се наричат времеви редове. В ежедневието си човек се сблъсква с много времеви редове. Например, средните дневни температури, количествата продадени горива през годината, стойностите на валутите на финансовите пазари и други. В почти всички области на човешката дейност се наблюдават такива явления и затова те са предмет на изучаване и прогнозиране.

Изкуствените невронни мрежи постигат голяма популярност в последните пет десетилетия. Основното им предимство е възможността да възпроизвеждат нелинейни зависимости с помощта на примерни данни. Приложение намират като инструмент за класификация, разпознаване на образи и прогнозиране. При най-разпространения вариант, изкуствените невронни мрежи представляват насочени тегловни графи. Организацията е на слоеве, като информацията се предава от входния слой към изходния слой. Най-често възлите между отделните слоеве са пълно свързани, което означава, че всеки възел е свързан с всички други възли от съседния слой. Организацията на броя слоеве и колко възли да има във всеки слой е обект на емпирично установяване и силно зависи от естеството на решаваната задача. Процесът на обучение най-често е с тренировъчни примери (обучение с учител) и целта е да се постигне такава оптимална стойност за теглата по ребрата на графа, така че изкуствената невронна мрежа максимално добре да извършва изчисленията, за които е предназначена. Веднъж обучени, изкуствените невронни мрежи са изключително бързо действащи. Тази тяхна характеристика ги прави особено желани в множество индустриални технически решения. Трудностите при употребата на изкуствени невронни мрежи са свързани с времето, необходимо за тяхното обучение. През десетилетията са разработени множество различни алгоритми за търсене на оптимални тегла в мрежата. Двете основни направления алгоритми са градиентни (точни числени алгоритми) и евристични (най-често стохастични с въведени емпирични правила). Ускоряването на процеса по обучение е основен проблем в практическата употреба на изкуствените невронни мрежи.

Цел на настоящия дисертационен труд е да се предложат хибридни алгоритми за ускоряване на обучението при изкуствени невронни мрежи от тип многослоен перцептрон за целите на прогнозирането на времеви редове. За реализиране на тази цел е необходимо да се изпълнят следните задачи:

*	Да се направи обзорен анализ и класификация на алгоритмите за обучение на изкуствени невронни мрежи от тип многослоен перцептрон;

*	Да се анализира възможността за комбиниране на различни алгоритми за реализиране на хибридно обучение на изкуствени невронни мрежи от тип многослоен перцептрон;

*	Да се предложат алгоритми за обучение на изкуствени невронни мрежи от тип многослоен перцептрон в разпределена среда;

*	Да се предложи подобрение с цел намаляване на времето за обучение на изкуствени невронни мрежи от тип многослоен перцептрон;

*	Да се предложи софтуерна архитектура за реализиране на мобилни разпределени изчисления за прогнозиране;

*	Да се направи програмна реализация на предложените хибридни алгоритми за обучение на изкуствени невронни мрежи от тип многослоен перцептрон с цел доказване на тяхната работоспособност;

*	Да се направи сравнителен анализ за ефективността на познатите алгоритми за обучение на изкуствени невронни мрежи от тип многослоен перцептрон.

Дисертационният труд е структуриран в увод, 4 глави, заключение, приноси, списък на публикациите, забелязани цитирания, декларация за оригиналност, библиография и 1 приложение.

В първа глава е направен обзорен анализ и класификация на широко използваните алгоритми за обучение на изкуствени невронни мрежи. Определени са предимствата и недостатъците на точните числени алгоритми и на евристичните алгоритми. Представени са възможностите за обучение на изкуствени невронни мрежи при последователни пресмятания, паралелни пресмятания и пресмятания в разпределена среда.

Във втора глава е изложена теорията за алгоритмите при обучението на изкуствени невронни мрежи от тип многослоен перцептрон. Предложени са модификации на някои от алгоритмите, които са приложими при прогнозирането на времеви редове. Приложените модификации се отнасят до: 1) определяне на теглата, чрез използване на генетичен алгоритъм, използващ операциите – селекция, кръстосване и мутация. Предложен и анализиран е нов оператор за селекция, основан на създаването на поколения при процедура за рекурсивно спускане. Изследвано е експериментално бързодействието на използваните евристични алгоритми; 2) апроксимиране на криви към множество точки – за инкрементална апроксимация на времеви редове се използват уравнение на права и ред от синус функции. Предложен е подход за пресмятане на коефициентите на синус функциите с оптимизатор, базиран на еволюция на разликите и рояк от частици; 3) обучението на изкуствена невронна мрежа – представен е разгърнат модел на обучението, което цели намиране на оптимални тегла за мрежа от тип трислоен перцептрон; 4) активационната функция в изкуствени невронни мрежи – предложена е алтернатива на производна за активационната функция в изкуствени невронни мрежи. Съществена особеност на предложената функция, е че показва обещаващи резултати по отношение на бързодействието и точността. За целите на численото тестване на предложените модификации е необходимо класифициране според честотата на гласуване за всеки потребител и процент на успех на всеки подаден глас. Решаването на тази конкретна задача е реализирано чрез използване на самоорганизиращи се карти на Кохонен.

В трета глава е представена софтуерна архитектура, позволяваща реализация на избрани алгоритми и предложените модификации. За реализацията на софтуерната архитектура е предложен обектно-ориентиран модел, релационен модел, комуникационните протоколи и графичен потребителски интерфейс. Всички те основаващи се на подходящи структури от данни.

В четвърта глава е изложен сравнителен анализ на някои точни числени и евристични алгоритми. Описани са проведените експерименти и получените резултати.  Анализирана е тяхната производителност и общо допусната грешка.

В заключението е направено обобщение на извършените изследвания. Посочени са също така и някои насоки за бъдещи изследвания, свързани с областта на приложението на изкуствените невронни мрежи.