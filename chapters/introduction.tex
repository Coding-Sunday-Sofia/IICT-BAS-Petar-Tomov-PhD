\addcontentsline{toc}{chapter}{Увод}
\chapter*{Увод}
\markboth{Увод}{}

Изкуствените невронни мрежи постигат изключително голяма популярност в последните пет десетилетия. Основното им предимство е възможността да възпроизвеждат нелинейни зависимости с помощта на примерни данни. Приложение намират като инструмент за класификация, разпознаване на образи и прогнозиране. При най-разпространеният вариант изкуствените невронни мрежи представляват насочени тегловни графи. Организацията е на слоеве, като информацията се предава от входния слой, към изходния слой. Най-често възлите между отделните слоеве са пълно свързани, което означава, че всеки възел е свързан с всички други възли от съседния слой. Организацията на броя слоеве и колко възли да има във всеки слой е обект на емпирично установяване и силно зависи от естеството на решаваната задача. Процесът на обучение най-често е с тренировъчни примери (обучение с учител) и целта е да се постигне такава оптимална стойност за теглата по ребрата на графа, така че изкуствената невронна мрежа максимално добре да извършва изчисленията за които е предназначена. 

\section*{Проблем}

Веднъж обучени изкуствените невронни мрежи са изключително бързо действащи. Тази тяхна характеристика ги прави особено желани в множество индустриални технически решения. Трудностите при употребата на изкуствени невронни мрежи са свързани с времето необходимо за тяхното обучение. През десетилетията са разработени множество различни алгоритми за търсене на оптимални тегла в мрежата. Двете основни направления алгоритми са градиентни (точни числени алгоритми) и евристични (най-често стохастични с въведени емпирични правила). Ускоряването на процеса по обучение е основен проблем в практическата употреба на изкуствените невронни мрежи.

\section*{Цел}

Основна цел на настоящия десертационен труд е предлагането на хибридни методи за ускоряване на обучението при изкуствени невронни мрежи от тип многослоен перцептрон. Многослойният перцептрон ще бъде приложен в прогнозирането на бъдещи стойности във финансови времеви редове. 

\section*{Задачи}

За постигане на целта поставена в настоящия дисертационен труд се поставят за решаване набор от задачи. Част от задачите са с теоретичен характер и покриват подбора на алгоритми за обучение, съчетаване на алгоритмите за обучение. Другата част от задачите са с приложен характер и са свързани с прилагане на подбраните алгоритми, подходящо съчетаване на различните алгоритми (хибридизация) и практическа реализация на цялостна система са прогнозиране. Набелязаните задачи са както следва:

* Обзор на алгоритмите за обучение на изкуствени неверонни мрежи от тип многослоен перцептрон;

* Подбор на алгоритми за хибридно обучение на изкуствени неверонни мрежи от тип многослоен перцептрон;

* Предлагане на алгоритми за обучение на изкуствени неверонни мрежи от тип многослоен перцептрон в разпределена среда;

* Предлагане на подобрения в алгоритмите, така че да се постигне скъсяване на времето за обучение на изкуствени неверонни мрежи от тип многослоен перцептрон;

* Програмна реализация на предложените хибридни алгоритми за обучение на изкуствени неверонни мрежи от тип многослоен перцептрон;

* Извършване на сравнителен анализ за ефективността от предложеното обучение на изкуствени неверонни мрежи от тип многослоен перцептрон;

\section*{Структура}

Дисртационният труд е организиран от въведение, четири глави, заключение и приложения. Изложението е в ??? страници, ?? фигури, ?? таблици, ?? листинги и ??? литературни източника в библиографията. По дисертационния труд има ?? публикации, като ?? от тях са доклади на международни конференции, а ?? са публикувани в национални издания с международна видимост. 

В първа глава е направен обзор на широко известните алгоритми за обучение на изкуствени невронни мрежи. Разгледани са предимствата и недостатъците на точните числени алгоритми и на евристичните алгоритми. Представени са възможностите за обучение на изкуствени невронни мрежи при последователни пресмятания, паралелни пресмятания и пресмятания в разпределена среда.

Във втора глава е изложена теорията за алгоритмите при обучението на изкуствени неверонни мрежи от тип многослоен перцептрон. Направен е подбор между точни числени алгоритми и евристични алгоритми. Направени са предложения за реализация на хибридни алгоритми в процеса на обучението на изкуствени неверонни мрежи от тип многослоен перцептрон. 

В трета глава е представена програмната реализация на подбраните алгоритми и хибридните модификации предложени в главата за теоретичната обосновка. Представени са използваните структури от данни, обектно-ориентиран модел, релационен модел, комуникационните протоколи и графичният потребителски интерфейс. 

В четвърта глава е изложен сравнителният анализ. Описани са проведените експерименти, получените резултати и е направен кратък анализ. Определя се ефективността на предложените подобрения в алгоритмите за обучение на изкуствени неверонни мрежи от тип многослоен перцептрон.

В заключението е направено обобщение на извършените изследвания, приложен е списък с публикациите по дисертационния труд, представен е списък със забелязани цитирания, дадени са насоки за бъдещо развитие, направено е обобщение на получените резултати.

