\documentclass[14pt,a4paper]{book}
%\degree{"Доктор"} 
%\thesistitle{ПРОГНОЗИРАНЕ НА ВРЕМЕВИ РЕДОВЕ С ИЗКУСТВЕНИ НЕВРОННИ МРЕЖИ}
%\author{\href{https://www.iict.bas.bg/ipdss/p-tomov-bg.html}{инж. Петър Росенов \textsc{Томов}}}
%\supervisor{\href{http://iinf.bas.bg/bg/monov.htm}{проф. д-р Владимир Василев \textsc{Монов}}}
%\addresses{ИИКТ-БАН, ул. "акад. Георги Бончев", блок 2, етаж 5, кабинет 514, град София 1113, България} 
%\subject{02.07.20 "Комуникационни мрежи и системи" \\ 5.3 "Комуникационна и компютърна техника" \\ 5 "Технически науки"} 
%\university{\href{http://bas.bg}{Българска академия на науките}}
%\faculty{\href{http://iict.bas.bg}{Институт по информационни и комуникационни технологии}} 
%\department{\href{http://iinf.bas.bg}{Моделиране и оптимизация}}

% Използване на български език.
\usepackage[english,bulgarian]{babel}
\usepackage[utf8]{inputenc}

% Използване на графика.
\usepackage[pdftex]{graphicx}

% Използване на PDF-и за кориците.
\usepackage{pdfpages}

% Използване на хедър и футър.
\usepackage{fancyhdr}

% Използване на кавички при цитиране.
\usepackage{dirtytalk}

% Използва се за създаване на азбучен указател.
\usepackage{imakeidx}

% Използва се за сензитивни хипер-връзки в самия документ.
\usepackage[pdftex, bookmarks, linktocpage]{hyperref}

% Използва се за листинги с програмен код.
\usepackage{listings}

% Заглавие.
\title{Прогнозиране на времеви редове с изкуствени невронни мрежи}

% Автор.
\author{инж. Петър Росенов Томов}

% Директория с изображения.
\graphicspath{{images/}}

% Избор на активен език.
\selectlanguage{bulgarian}

% Текстове за декорация на страницата в горната и долната част.
\pagestyle{fancy}
\fancyhf{}
\fancyhead[LE,RO]{\thepage}
\fancyhead[RE]{Прогнозиране на времеви редове с изкуствени невронни мрежи}
\fancyhead[LO]{\thechapter}
\fancyfoot[LE,RO]{Петър Томов - ИИКТ-БАН - София - 2021}

% Дебелина на разделителните линии.
\renewcommand{\headrulewidth}{2pt}
\renewcommand{\footrulewidth}{1pt}

% Генериране на азбучен указател.
\onecolumn
\makeindex[columns=2, title=Азбучен указател, intoc]

% Подменя думата използван а за ноемрация на фрагментите програмен код.
\renewcommand{\lstlistingname}{Листинг}

% Смяна на названието за списъка от листингите.
\renewcommand{\lstlistlistingname}{Списък на листингите}

% Определя характеристиките на листигните за програмния код.
\lstset{backgroundcolor=\color{gray!30}, breaklines=true, language=r, frame=single}

% Начало на документа.
\begin{document}

% Предна корица.
%\includepdf[pages={1}]{covers/front}
\thispagestyle{empty}

% Номериране на страниците със служебна информация.
\pagenumbering{roman}
\setcounter{page}{1}

% Таблица на съдържанието.
\addcontentsline{toc}{chapter}{Съдържание}
\tableofcontents\newpage

% Списък с фигурите.
\addcontentsline{toc}{chapter}{Списък на фигурите}
\listoffigures\newpage

% Списък с таблиците.
\addcontentsline{toc}{chapter}{Списък на таблиците}
\listoftables\newpage

% Списък с листингите.
\addcontentsline{toc}{chapter}{Списък на листингите}
\lstlistoflistings\newpage

% Номериране на страниците с основното изложение.
\pagenumbering{arabic}
\setcounter{page}{1}

% Отделните глави са в отделни файлове.
%\addcontentsline{toc}{chapter}{Въведеие}
\chapter*{Въведеие}\newpage
%\chapter{Обзор на алгоритмите за обучение на изкуствени невронни мрежи}

В началото на 70-те години на XX век се предлагат линейни модели за прогнозиране на времеви редове – авторегресионен (Autoregressive) и плъзгащи средни (Moving Averages) \cite{Tealab-01}. При авторегресионния модел се възприема, че прогнозната стойност е линейна комбинация от стойностите в миналото. При плъзгащите средни прогнозната стойност е функция на случайни смущения - смущения, които са повлияли на времевия ред. Между двата модела е възможна комбинация под формата на авторегресионна интегрирана плъзгаща средна (Auto-Regressive Integrated Moving Average) \cite{Khashei-01}. Изкуствените неверонни мрежи се оказват един от подходите да се въведе нелинейност в моделите.

\newpage
%\chapter{Предложение за алгоритми при обучение на изкуствени невронни мрежи}

\section{Бърз прототип на LibreOffice Calc с Differential Evolution и Particle Swarm Optimization}


\newpage
%\chapter{Програмна реализация на система за прогнозиране}
\newpage
%\chapter{Сравнителен анализ – експерименти и резултати}
\newpage
%\addcontentsline{toc}{chapter}{Заключение}
\chapter*{Заключение}

\newpage
\subsection*{Резюме на получените резултати}

\newpage
\subsection*{Насоки за бъдещи изследвания}

\newpage
\subsection*{Публикации по темата на дисертационния труд}

\newpage
\subsection*{Забелязани цитирания}

\newpage
\subsection*{Декларация за оригиналност на резултатите}

\vspace{1cm}

Декларирам, че дисертацията съдържа оригинални резултати, получени при проведени от мен, научни изследвания с подкрепата и съдействието на научния ми ръководител.

Резултатите, които са получени, описани и/или публикувани от други учени, са коректно и подробно цитирани в библиографията.

Настоящият дисертационен труд не е прилаган за придобиване на научна степен в друго висше училище, университет или научен институт.

\vspace{2cm}

\begin{tabular}{ c c c c }
Дата: & .......................................... & Подпис: & .......................................... \\ 
& гр. София & & / Петър Томов / \\  
\end{tabular}
\newpage

% Списък с използвана литература и източници на информация.
\addcontentsline{toc}{chapter}{Библиография}
%\addcontentsline{toc}{chapter}{Библиография}
\chapter*{Библиография}

\begin{thebibliography}{99.}

\bibitem{Bontempi-01} Bontempi, G., Ben Taieb, S., Le Borgne Y.: Machine Learning Strategies for Time Series Forecasting. Lecture Notes in Business Information Processing, vol 138, 62--77, 2013. ISBN 978-3-642-36317-7 DOI 10.1007/978-3-642-36318-4\_3

\bibitem{Cao-01} Cao, L., Tay, F.: Support vector machine with adaptive parameters in financial time series forecasting. IEEE Transactions on Neural Networks, vol. 14, no. 6, 1506--1518, 2003. ISSN 1045-9227 DOI 10.1109/TNN.2003.820556

\bibitem{Gooijer-01} Gooijer, J., Hyndman, R.: 25 years of time series forecasting. International Journal of Forecasting, vol. 22, no. 3, 443--473, 2006. ISSN 0169-2070 DOI 10.1016/j.ijforecast.2006.01.001

\bibitem{Jain-01} Jain, A., Kumar, A.: Hybrid neural network models for hydrologic time series forecasting. Applied Soft Computing, vol. 7, no. 2, 585--592, 2007. ISSN 1568-4946 DOI 10.1016/j.asoc.2006.03.002

\bibitem{Khashei-01} Khashei, M., Bijari, M.: An artificial neural network (p,d,q) model for timeseries forecasting. Expert Systems with Applications, vol. 37, no. 1, 479--489, 2010. ISSN 0957-4174 DOI 10.1016/j.eswa.2009.05.044

\bibitem{Kyoung-jae-01} Kyoung-jae, K.: Financial time series forecasting using support vector machines. Neurocomputing, vol. 55, no. 1, 307--319, 2003. ISSN 0925-2312 DOI 10.1016/S0925-2312(03)00372-2

\bibitem{Tang-01} Tang, Z., de Almeida, C., Fishwick, P.: Time series forecasting using neural networks vs Box-Jenkins methodology. SIMULATION, vol. 57, no. 5, 303-310, 1991. ISSN 0037-5497 DOI 10.1177/003754979105700508

\bibitem{Tay-01} Tay, F., Cao, L.: Application of support vector machines in financial time series forecasting. Omega, vol. 29, no. 4, 309--317, 2001. ISSN 0305-0483 DOI 10.1016/S0305-0483(01)00026-3

\bibitem{Tealab-01} Tealab, A.: Time series forecasting using artificial neural networks methodologies: A systematic review. Future Computing and Informatics Journal, vol. 3, no. 2, 334--340, 2018. ISSN 2314-7288 DOI 10.1016/j.fcij.2018.10.003

\bibitem{Zhang-03} Zhang, G: An investigation of neural networks for linear time-series forecasting. Computers and Operations Research, vol. 28, no. 12, 1183-1202, 2001. ISSN 0305-0548 DOI 10.1016/S0305-0548(00)00033-2

\bibitem{Zhang-01} Zhang, G.: Time series forecasting using a hybrid ARIMA and neural network model. Neurocomputing, vol. 50, 159--175, 2003. ISSN 0925-2312 DOI 10.1016/S0925-2312(01)00702-0

\bibitem{Zhang-02} Zhang, G., Qi, M.: Neural network forecasting for seasonal and trend time series. European Journal of Operational Research, vol. 160, no. 2, 501--514, 2005. ISSN 0377-2217 DOI 10.1016/j.ejor.2003.08.037

\end{thebibliography}
\newpage

% Азбучен указател на използваните термини.
\printindex

% Задна корица.
%\includepdf[pages=-]{covers/back}

\end{document}
