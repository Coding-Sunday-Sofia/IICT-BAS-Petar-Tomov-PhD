\documentclass[12pt,a4paper,openany]{book}
\usepackage{layout}
%\degree{"Доктор"} 
%\thesistitle{ПРОГНОЗИРАНЕ НА ВРЕМЕВИ РЕДОВЕ С ИЗКУСТВЕНИ НЕВРОННИ МРЕЖИ}
%\author{\href{https://www.iict.bas.bg/ipdss/p-tomov-bg.html}{инж. Петър Росенов \textsc{Томов}}}
%\supervisor{\href{http://iinf.bas.bg/bg/monov.htm}{проф. д-р Владимир Василев \textsc{Монов}}}
%\addresses{ИИКТ-БАН, ул. "акад. Георги Бончев", блок 2, етаж 5, кабинет 514, град София 1113, България} 
%\subject{02.07.20 "Комуникационни мрежи и системи" \\ 5.3 "Комуникационна и компютърна техника" \\ 5 "Технически науки"} 
%\university{\href{http://bas.bg}{Българска академия на науките}}
%\faculty{\href{http://iict.bas.bg}{Институт по информационни и комуникационни технологии}} 
%\department{\href{http://iinf.bas.bg}{Моделиране и оптимизация}}

% Добавя възможност за сензитивни хипер-връзки в самия документ.
\usepackage[pdftex, bookmarks, linktocpage]{hyperref}

% Команда с множество опции за настройка на поведението на пакета hyperref, с най-полезната опция - кирилизация на заглавията от Bookmarks в Acrobat.
\hypersetup{unicode=true, colorlinks=true, linkcolor=black, citecolor=black, urlcolor=black}

%Set margins
\usepackage[left=2.5cm,right=2.5cm,top=3.5cm,bottom=3.5cm]{geometry}

%tab first paragraph
\usepackage{indentfirst}

\usepackage{makecell}
\usepackage{tabularx,colortbl}

% Използване на български език.
\usepackage[T2A,T1]{fontenc}
\usepackage[utf8]{inputenc}
\usepackage[english,bulgarian]{babel}

% Използва се за групиране на изображения.
\usepackage{subcaption}

% Използване на графика.
\usepackage[pdftex]{graphicx}

% Използване на по прецизни позиции за изображенията.
\usepackage{float}

% Използване на PDF-и за кориците.
\usepackage{pdfpages}

% Използване на хедър и футър.
\usepackage{fancyhdr}

% Използване на кавички при цитиране.
\usepackage{dirtytalk}

% Използва се за създаване на азбучен указател.
\usepackage{imakeidx}

% Използва се за листинги с програмен код.
\usepackage{listings}

% Използва се за оцветяване на клетките в таблиците.
\usepackage{xcolor,colortbl}

% Използва се за многоредови коментари.
\usepackage{verbatim}

% Използвасе за междуредово разстояние.
\usepackage{lipsum}
\usepackage{setspace}

% Използва се за таблици, които да са на повече от една страница.
\usepackage{longtable}

% Заглавие.
\title{Прогнозиране на времеви редове с изкуствени невронни мрежи}
\makeatletter
\let\inserttitle\@title
\makeatother

% Автор.
\author{инж. Петър Росенов Томов}

% Директория с изображения.
\graphicspath{{images/}}

% Избор на активен език.
\selectlanguage{bulgarian}

% Текстове за декорация на страницата в горната и долната част.
\pagestyle{fancy}
\fancyhf{}
\fancyhead[LE,RO]{\thepage}
\fancyhead[RE]{\tiny\MakeUppercase{\inserttitle}}
\fancyhead[LO]{\tiny\leftmark}
\fancyfoot[LE,RO]{Петър Томов - ИИКТ-БАН - София - 2022}

% Дебелина на разделителните линии.
\renewcommand{\headrulewidth}{2pt}
\renewcommand{\footrulewidth}{1pt}

% Генериране на азбучен указател.
\onecolumn
\makeindex[columns=2, title=Азбучен указател, intoc]

% Подменя думата използван а за ноемрация на фрагментите програмен код.
\renewcommand{\lstlistingname}{Листинг}

% Смяна на названието за списъка от листингите.
\renewcommand{\lstlistlistingname}{Списък на листингите}

% Определя характеристиките на листигните за програмния код.
\lstset{backgroundcolor=\color{gray!30}, breaklines=true, language=r, frame=single}

% Разстояние от ред и половина.
\linespread{1.5}

% Начало на документа.
\begin{document}

%Принт отстъпите на страницата
%\layout 

% Предна корица.
\includepdf[pages={1}]{covers/front}
\thispagestyle{empty}

% Номериране на страниците със служебна информация.
\pagenumbering{roman}
\setcounter{page}{1}

% Таблица на съдържанието.
\addcontentsline{toc}{chapter}{Съдържание}
\tableofcontents

% Списък с абревиатурите и съкращенията.
\addcontentsline{toc}{chapter}{Списък на съкращенията}
\chapter*{Списък на съкращенията}

\begin{longtable}{ | p{3.75cm} | c | p{3.75cm} | c | }
\hline
\cellcolor{gray!15}Чуждоезичен термин & \cellcolor{gray!15}Съкращение & \cellcolor{gray!15}Български термин & \cellcolor{gray!15}Съкращение \\ [0.05ex] 
\hline
\hline
Artificial Neural Network & ANN & Изкуствена невронна мрежа & ИНМ \\  
\hline
Back Propagation  & BP & Обратно разрпостранение на грешката & ОРГ \\  
\hline
Differential Evolution & DE & Еволюция на разликите & ЕР \\ 
\hline
Donated Distributed Computing & DDC & Дарени разпределени изчисления & ДРИ \\  
\hline
Evolutionary Algorithms & EAs & Еволюционни алгоритми & ЕА \\  
\hline
Evolution Strategy & ES & Еволюционна стратегия & ЕС \\  
\hline
Genetic Algorithms & GAs & Генетични алгоритми & ГА \\  
\hline
Learning Rate & LR & Норма за обучение & НО \\  
\hline
Moving Averages & MA & Плъзгащи средни & ПС \\  
\hline
Machine Learning & ML & Машинно самообучение & МС \\  
\hline
Multi-Layer Perceptron & MLP & Многослоен перцептрон & МП \\  
\hline
Multiobjective Evolutionary Algorithms & MOEA & Многокритериални еволюционни алгоритми & МЕА \\  
\hline
Operating System & OS & Операционна система & ОС \\  
\hline
Particle Swarm Optimization & PSO & Оптимизация с рояци от частици & ОРЧ \\  
\hline
Patterns Recognition & PR & Разпознаване на образи & РО \\  
\hline
Probabilistic Neural Networks & PNN & Вероятностни невронни мрежи & ВНМ \\  
\hline
Support Vector Machines & SVM & Машини с поддържащи вектори & МПВ \\  
\hline
Time Series & TS & Времеви редове & ВР \\  
\hline
Volunteer Distributed Computing & VDC & Доброволчески разпределени изчисления & ДРИ \\  
\hline
\end{longtable}

% Списък с фигурите.
\addcontentsline{toc}{chapter}{Списък на фигурите}
\listoffigures

% Списък с таблиците.
\addcontentsline{toc}{chapter}{Списък на таблиците}
\listoftables

% Списък с листингите.
\addcontentsline{toc}{chapter}{Списък на листингите}
\lstlistoflistings

\newpage

% Номериране на страниците с основното изложение.
\pagenumbering{arabic}
\setcounter{page}{1}

% Увод.
\addcontentsline{toc}{chapter}{Въведеие}
\chapter*{Въведеие}

% Отделните глави са в отделни файлове.
\chapter{Обзор на алгоритмите за обучение на изкуствени невронни мрежи}

В началото на 70-те години на XX век се предлагат линейни модели за прогнозиране на времеви редове – авторегресионен (Autoregressive) и плъзгащи средни (Moving Averages) \cite{Tealab-01}. При авторегресионния модел се възприема, че прогнозната стойност е линейна комбинация от стойностите в миналото. При плъзгащите средни прогнозната стойност е функция на случайни смущения - смущения, които са повлияли на времевия ред. Между двата модела е възможна комбинация под формата на авторегресионна интегрирана плъзгаща средна (Auto-Regressive Integrated Moving Average) \cite{Khashei-01}. Изкуствените неверонни мрежи се оказват един от подходите да се въведе нелинейност в моделите.


\chapter{Предложение за алгоритми при обучение на изкуствени невронни мрежи}

\section{Бърз прототип на LibreOffice Calc с Differential Evolution и Particle Swarm Optimization}



\chapter{Програмна реализация на система за прогнозиране}

\chapter{Сравнителен анализ – експерименти и резултати}


% Заключение.
\addcontentsline{toc}{chapter}{Заключение}
\chapter*{Заключение}

\newpage
\subsection*{Резюме на получените резултати}

\newpage
\subsection*{Насоки за бъдещи изследвания}

\newpage
\subsection*{Публикации по темата на дисертационния труд}

\newpage
\subsection*{Забелязани цитирания}

\newpage
\subsection*{Декларация за оригиналност на резултатите}

\vspace{1cm}

Декларирам, че дисертацията съдържа оригинални резултати, получени при проведени от мен, научни изследвания с подкрепата и съдействието на научния ми ръководител.

Резултатите, които са получени, описани и/или публикувани от други учени, са коректно и подробно цитирани в библиографията.

Настоящият дисертационен труд не е прилаган за придобиване на научна степен в друго висше училище, университет или научен институт.

\vspace{2cm}

\begin{tabular}{ c c c c }
Дата: & .......................................... & Подпис: & .......................................... \\ 
& гр. София & & / Петър Томов / \\  
\end{tabular}


% Списък с използвана литература и източници на информация.
\addcontentsline{toc}{chapter}{Библиография}
\chapter*{Библиография}

\begin{thebibliography}{99.}

\bibitem{Bontempi-01} Bontempi, G., Ben Taieb, S., Le Borgne Y.: Machine Learning Strategies for Time Series Forecasting. Lecture Notes in Business Information Processing, vol 138, 62--77, 2013. ISBN 978-3-642-36317-7 DOI 10.1007/978-3-642-36318-4\_3

\bibitem{Cao-01} Cao, L., Tay, F.: Support vector machine with adaptive parameters in financial time series forecasting. IEEE Transactions on Neural Networks, vol. 14, no. 6, 1506--1518, 2003. ISSN 1045-9227 DOI 10.1109/TNN.2003.820556

\bibitem{Gooijer-01} Gooijer, J., Hyndman, R.: 25 years of time series forecasting. International Journal of Forecasting, vol. 22, no. 3, 443--473, 2006. ISSN 0169-2070 DOI 10.1016/j.ijforecast.2006.01.001

\bibitem{Jain-01} Jain, A., Kumar, A.: Hybrid neural network models for hydrologic time series forecasting. Applied Soft Computing, vol. 7, no. 2, 585--592, 2007. ISSN 1568-4946 DOI 10.1016/j.asoc.2006.03.002

\bibitem{Khashei-01} Khashei, M., Bijari, M.: An artificial neural network (p,d,q) model for timeseries forecasting. Expert Systems with Applications, vol. 37, no. 1, 479--489, 2010. ISSN 0957-4174 DOI 10.1016/j.eswa.2009.05.044

\bibitem{Kyoung-jae-01} Kyoung-jae, K.: Financial time series forecasting using support vector machines. Neurocomputing, vol. 55, no. 1, 307--319, 2003. ISSN 0925-2312 DOI 10.1016/S0925-2312(03)00372-2

\bibitem{Tang-01} Tang, Z., de Almeida, C., Fishwick, P.: Time series forecasting using neural networks vs Box-Jenkins methodology. SIMULATION, vol. 57, no. 5, 303-310, 1991. ISSN 0037-5497 DOI 10.1177/003754979105700508

\bibitem{Tay-01} Tay, F., Cao, L.: Application of support vector machines in financial time series forecasting. Omega, vol. 29, no. 4, 309--317, 2001. ISSN 0305-0483 DOI 10.1016/S0305-0483(01)00026-3

\bibitem{Tealab-01} Tealab, A.: Time series forecasting using artificial neural networks methodologies: A systematic review. Future Computing and Informatics Journal, vol. 3, no. 2, 334--340, 2018. ISSN 2314-7288 DOI 10.1016/j.fcij.2018.10.003

\bibitem{Zhang-03} Zhang, G: An investigation of neural networks for linear time-series forecasting. Computers and Operations Research, vol. 28, no. 12, 1183-1202, 2001. ISSN 0305-0548 DOI 10.1016/S0305-0548(00)00033-2

\bibitem{Zhang-01} Zhang, G.: Time series forecasting using a hybrid ARIMA and neural network model. Neurocomputing, vol. 50, 159--175, 2003. ISSN 0925-2312 DOI 10.1016/S0925-2312(01)00702-0

\bibitem{Zhang-02} Zhang, G., Qi, M.: Neural network forecasting for seasonal and trend time series. European Journal of Operational Research, vol. 160, no. 2, 501--514, 2005. ISSN 0377-2217 DOI 10.1016/j.ejor.2003.08.037

\end{thebibliography}


% Азбучен указател на използваните термини.
\printindex

% Приложения.
\footnotesize
\begin{spacing}{0.8}
\addcontentsline{toc}{chapter}{Приложение А - програмен код}
\chapter*{Приложение А - програмен код}
\markboth{ПРИЛОЖЕНИЕ А - ПРОГРАМЕН КОД}{}

Представеният програмен код представлява самостоятелно приложение за мобилни устройства, работещи с операционната система Android OS. Мобилното приложение комуникира с отдалечен сървър, разработката на който не е част от настоящия дисертационен труд. Програмния код включва графичен потребителски нитерфейс, работа във фонов режим, хибриден алгоритъм за обучение на трислоен перцептрон. Програмният код работи на мобилни устройства с операционна система Android OS, поне версия 11. Приложението зарежда времеви редове от отдалечен сървър. Времевите редове са предимно финансови, но на сървъра може да се зареждат и други видове времеви редове. Времевите редове се декомпозират на тренировъчни примери и се обучава изкуствената невронна мрежа. 

\vspace*{5mm}

\textbf{\underline{ForecastDatabaseHelper.java}}
\begin{verbatim}
package eu.veldsoft.vitosha.trade;

import android.content.Context;
import android.database.sqlite.SQLiteDatabase;
import android.database.sqlite.SQLiteOpenHelper;
import android.provider.BaseColumns;

/**
 * Database helper class.
 */
class ForecastDatabaseHelper extends SQLiteOpenHelper {
    /**
     * Database integer version.
     */
    public static final int DATABASE_VERSION = 1;
    /**
     * Database file name.
     */
    public static final String DATABASE_NAME = "Forecast.db";
    /**
     * Drop rates table SQL pattern.
     */
    static final String SQL_DELETE_RATES = "DROP TABLE IF EXISTS "
            + RatesColumns.TABLE_NAME;
    /**
     * Drop rates table SQL pattern.
     */
    static final String SQL_DELETE_ANNS = "DROP TABLE IF EXISTS "
            + ANNsColumns.TABLE_NAME;
    /**
     * Create rates table SQL patter.
     */
    private static final String SQL_CREATE_RATES = "CREATE TABLE "
            + RatesColumns.TABLE_NAME + " (" + RatesColumns._ID
            + " INTEGER NOT NULL," + RatesColumns.COLUMN_NAME_SYMBOL
            + " TEXT, " + RatesColumns.COLUMN_NAME_PERIOD
            + " INTEGER, " + RatesColumns.COLUMN_NAME_TIME + " TEXT, "
            + RatesColumns.COLUMN_NAME_OPEN + " TEXT, "
            + RatesColumns.COLUMN_NAME_LOW + " TEXT, "
            + RatesColumns.COLUMN_NAME_HIGH + " TEXT, "
            + RatesColumns.COLUMN_NAME_CLOSE + " TEXT, "
            + RatesColumns.COLUMN_NAME_VOLUME + " TEXT PRIMARY KEY ("
            + RatesColumns.COLUMN_NAME_SYMBOL + ", "
            + RatesColumns.COLUMN_NAME_PERIOD + "))";

    /**
     * Create ANNs table SQL patter.
     */
    private static final String SQL_CREATE_ANNS = "CREATE TABLE "
            + ANNsColumns.TABLE_NAME + " (" + ANNsColumns._ID
            + " INTEGER PRIMARY KEY," + ANNsColumns.COLUMN_NAME_SYMBOL
            + " TEXT, " + ANNsColumns.COLUMN_NAME_PERIOD
            + " INTEGER, " + ANNsColumns.COLUMN_NAME_NEURONS + " TEXT, "
            + ANNsColumns.COLUMN_NAME_ACTIVITIES
            + " TEXT, " + ANNsColumns.COLUMN_NAME_WEIGHTS
            + " TEXT, FOREIGN KEY (" + ANNsColumns.COLUMN_NAME_SYMBOL
            + ") REFERENCES " + RatesColumns.TABLE_NAME
            + "(" + RatesColumns.COLUMN_NAME_SYMBOL + "), FOREIGN KEY ("
            + ANNsColumns.COLUMN_NAME_PERIOD + ") REFERENCES "
            + RatesColumns.TABLE_NAME + "("
            + RatesColumns.COLUMN_NAME_PERIOD + "))";

    /**
     * Constructor.
     *
     * @param context Context of database helper usage.
     */
    public ForecastDatabaseHelper(Context context) {
        super(context, DATABASE_NAME, null, DATABASE_VERSION);
    }

    /**
     * {@inheritDoc}
     */
    @Override
    public void onCreate(SQLiteDatabase db) {
        db.execSQL(SQL_CREATE_RATES);
        db.execSQL(SQL_CREATE_ANNS);
    }

    /**
     * {@inheritDoc}
     */
    @Override
    public void onUpgrade(SQLiteDatabase db, int oldVersion, int newVersion) {
        db.execSQL(SQL_DELETE_ANNS);
        db.execSQL(SQL_DELETE_RATES);
        onCreate(db);
    }

    /**
     * Rates table columns description class.
     */
    public static abstract class RatesColumns implements BaseColumns {
        public static final String TABLE_NAME = "rates";
        public static final String COLUMN_NAME_SYMBOL = "symbol";
        public static final String COLUMN_NAME_PERIOD = "period";
        public static final String COLUMN_NAME_TIME = "time";
        public static final String COLUMN_NAME_OPEN = "open";
        public static final String COLUMN_NAME_LOW = "low";
        public static final String COLUMN_NAME_HIGH = "high";
        public static final String COLUMN_NAME_CLOSE = "close";
        public static final String COLUMN_NAME_VOLUME = "volume";
    }

    /**
     * ANN table columns description class.
     */
    public static abstract class ANNsColumns implements BaseColumns {
        public static final String TABLE_NAME = "anns";
        public static final String COLUMN_NAME_SYMBOL = "symbol";
        public static final String COLUMN_NAME_PERIOD = "period";
        public static final String COLUMN_NAME_NEURONS = "neurons";
        public static final String COLUMN_NAME_ACTIVITIES = "activities";
        public static final String COLUMN_NAME_WEIGHTS = "weights";
    }
}
\end{verbatim}

\textbf{\underline{ProgressReportingWallpaperConfigureActivity.java}}
\begin{verbatim}
package eu.veldsoft.vitosha.trade;

import android.app.WallpaperManager;
import android.content.ComponentName;
import android.content.Intent;
import android.content.SharedPreferences;
import android.os.Bundle;
import android.preference.PreferenceActivity;
import android.preference.PreferenceManager;

/**
 * Options screen.
 */
public class ProgressReportingWallpaperConfigureActivity extends PreferenceActivity {

    /**
     * {@inheritDoc}
     */
    @Override
    protected void onCreate(Bundle savedInstanceState) {
        super.onCreate(savedInstanceState);
        addPreferencesFromResource(R.layout.wallpaper_configure);
    }

    /**
     * {@inheritDoc}
     */
    @Override
    protected void onPause() {
        super.onPause();
        SharedPreferences preferences = PreferenceManager
                .getDefaultSharedPreferences(ProgressReportingWallpaperConfigureActivity.this);

        /*
         * Remove our wallpaper.
         */
        if (preferences.getBoolean("set_wallpaper", false) == false) {
            stopService(new Intent(ProgressReportingWallpaperConfigureActivity.this,
                    ProgressReportingWallpaperService.class));
            startActivity(new Intent(WallpaperManager.ACTION_LIVE_WALLPAPER_CHOOSER));
            ProgressReportingWallpaperConfigureActivity.this.finish();

            return;
        }

        /*
         * Run wallpaper service.
         */
        Intent intent = new Intent(WallpaperManager.ACTION_CHANGE_LIVE_WALLPAPER);
        intent.putExtra(WallpaperManager.EXTRA_LIVE_WALLPAPER_COMPONENT,
                new ComponentName(ProgressReportingWallpaperConfigureActivity.this,
                        ProgressReportingWallpaperService.class));
        startActivity(intent);
        ProgressReportingWallpaperConfigureActivity.this.finish();
    }
}
\end{verbatim}

\textbf{\underline{ProgressReportingWallpaperService.java}}
\begin{verbatim}
package eu.veldsoft.vitosha.trade;

import android.content.SharedPreferences;
import android.graphics.Bitmap;
import android.graphics.BitmapFactory;
import android.graphics.Canvas;
import android.graphics.Color;
import android.graphics.Paint;
import android.graphics.Rect;
import android.os.Handler;
import android.preference.PreferenceManager;
import android.service.wallpaper.WallpaperService;
import android.view.SurfaceHolder;

import java.util.Calendar;
import java.util.Random;

import eu.veldsoft.vitosha.trade.communication.HttpHelper;
import eu.veldsoft.vitosha.trade.communication.TimePeriod;
import eu.veldsoft.vitosha.trade.dummy.InputData;
import eu.veldsoft.vitosha.trade.engine.Predictor;

/**
 * Background calculation unit.
 */
public class ProgressReportingWallpaperService extends WallpaperService {

    /**
     * Pseudo-random number generator.
     */
    private static final Random PRNG = new Random();

    /**
     * Space between visual spots in pixels.
     */
    private static final int GAP_BETWEEN_PANELS = 10;

    /**
     * Default training time delay.
     */
    private static final long DEFAULT_DELAY = 86400000L;

    //TODO Images should be loaded from a remote image server.

    /**
     * Identifiers for the background resources images to be used as background.
     */
    private static final int[] IMAGES_IDS = {
            R.drawable.vitosha_mountain_dimitar_petarchev_001,
            R.drawable.vitosha_mountain_dimitar_petarchev_002,
            R.drawable.vitosha_mountain_dimitar_petarchev_003,
            R.drawable.vitosha_mountain_dimitar_petarchev_004,
    };

    // TODO Put all colors in the settings dialog.

    /**
     * Panel background color in order to be a part transparent from the real background.
     */
    private static final int PANEL_BACKGROUND_COLOR =
            Color.argb(63, 0, 0, 0);

    /**
     * Text color to be used in panels.
     */
    private static final int PANEL_TEXT_COLOR =
            Color.argb(95, 255, 255, 255);

    /**
     * Time delay between neural network trainings.
     */
    private static long delay = 0;

    /**
     * Visible surface width.
     */
    private static int screenWidth = 0;

    /**
     * Visible surface height.
     */
    private static int screenHeight = 0;

    /**
     * Wallpaper visibility flag.
     */
    private static boolean visible = false;

    /**
     * List of information panels rectangles information.
     */
    private static Rect[] panels = {new Rect(), new Rect(), new Rect()};

    /**
     * Forecasting object.
     */
    private static Predictor predictor = new Predictor();

    /**
     * Initialize common class members.
     */
    private void initialize() {
        /*
         * Load ANN structure and time series data from the remote server.
         */
        HttpHelper helper = new HttpHelper(PreferenceManager
                .getDefaultSharedPreferences(
                        ProgressReportingWallpaperService.this).
                        getString("server_url", "localhost"));

        if (helper.load() == false) {
            // TODO Use local data if the remote server is not available.
        }

        predictor.initialize();
    }

    /**
     * {@inheritDoc}
     */
    @Override
    public void onCreate() {
        super.onCreate();

        initialize();
    }

    /**
     * {@inheritDoc}
     */
    @Override
    public Engine onCreateEngine() {
        return new WallpaperEngine();
    }

    /**
     * Wallpaper engine class.
     */
    private class WallpaperEngine extends Engine {

        /**
         * Thread handler.
         */
        private final Handler handler = new Handler();

        /**
         * Paint object.
         */
        private final Paint paint = new Paint();

        /**
         * Neural network training cycle thread.
         */
        private final Runnable trainer = new Runnable() {
            @Override
            public void run() {
                predictor.predict();
                draw();
                predictor.train();
            }
        };

        /**
         * Constructor without parameters.
         */
        public WallpaperEngine() {
            super();
            handler.post(trainer);
        }

        /**
         * Common drawing procedure.
         */
        private void draw() {
            SurfaceHolder holder = getSurfaceHolder();
            Canvas canvas = null;

            try {
                canvas = holder.lockCanvas();

                if (canvas != null) {
                    drawBackground(canvas);
                    drawPanels(canvas);
                    drawCurrencyPairInfo(canvas);
                    drawForecast(canvas);
                    drawAnn(canvas);
                }
            } finally {
                if (canvas != null) {
                    holder.unlockCanvasAndPost(canvas);
                }
            }

            handler.removeCallbacks(trainer);
            if (visible == true) {
                handler.postDelayed(trainer, delay);
            }
        }

        /**
         * Background drawing procedure.
         *
         * @param canvas Canvas object for background.
         */
        private void drawBackground(Canvas canvas) {
            // TODO Images should be loaded from an image server.
            /*
             * Change picture according the day in the year.
             */
            Bitmap image = BitmapFactory.decodeResource(
                    ProgressReportingWallpaperService.this.getResources(),
                    IMAGES_IDS[Calendar.getInstance().
                            get(Calendar.DAY_OF_YEAR) % IMAGES_IDS.length]);

            /*
             * Select random top-left corner for image clip.
             */
            int left = PRNG.nextInt(image.getWidth() - screenWidth);
            int top = PRNG.nextInt(image.getHeight() - screenHeight);

            /*
             * Clip part of the image.
             */
            canvas.drawBitmap(image, new Rect(left, top,
                            left + screenWidth - 1,
                            top + screenHeight - 1),
                    new Rect(0, 0, screenWidth - 1,
                            screenHeight - 1), null);
        }

        /**
         * Panels drawing procedure.
         *
         * @param canvas Canvas object for panels drawing.
         */
        private void drawPanels(Canvas canvas) {
            /*
             * Panels.
             */
            paint.setColor(PANEL_BACKGROUND_COLOR);
            for (Rect rectangle : panels) {
                canvas.drawRect(rectangle, paint);
            }
        }

        /**
         * Currency pair info drawing procedure.
         *
         * @param canvas Canvas object for currency pair info drawing.
         */
        private void drawCurrencyPairInfo(Canvas canvas) {
            /*
             * Time series info.
             */
            int textSize = (panels[0].bottom - panels[0].top) / 5;
            paint.setTextSize(textSize);
            paint.setColor(PANEL_TEXT_COLOR);
            canvas.drawText("" + InputData.SYMBOL,
                    GAP_BETWEEN_PANELS + panels[0].left,
                    GAP_BETWEEN_PANELS + panels[0].top + textSize, paint);
            canvas.drawText("" + TimePeriod.value(InputData.PERIOD),
                    GAP_BETWEEN_PANELS + panels[0].left,
                    GAP_BETWEEN_PANELS + panels[0].top + 2 * textSize, paint);
        }

        /**
         * Forecast drawing procedure.
         *
         * @param canvas Canvas object for forecast drawing.
         */
        private void drawForecast(Canvas canvas) {
            int width = panels[1].right - panels[1].left;
            int height = panels[1].bottom - panels[1].top;
            int stride = width;

            int[] pixels = new int[width * height];
            predictor.drawForecast(pixels, width, height);
            Bitmap bitmap = Bitmap.createBitmap(pixels, 0, stride, width, height, Bitmap.Config.ARGB_8888);
            canvas.drawBitmap(bitmap, new Rect(0,0,width,height), panels[1], paint);
        }

        /**
         * Neural network drawing procedure.
         *
         * @param canvas Canvas object for neural network drawing.
         */
        private void drawAnn(Canvas canvas) {
            int width = panels[2].right - panels[2].left;
            int height = panels[2].bottom - panels[2].top;
            int stride = width;

            int[] pixels = new int[width * height];
            predictor.drawAnn(pixels, width, height);
            Bitmap bitmap = Bitmap.createBitmap(pixels, 0, stride, width, height, Bitmap.Config.ARGB_8888);
            canvas.drawBitmap(bitmap, new Rect(0,0,width,height), panels[2], paint);
        }

        /**
         * {@inheritDoc}
         */
        @Override
        public void onVisibilityChanged(boolean visible) {
            ProgressReportingWallpaperService.visible = visible;

            /*
             * Do calculations only if the wallpaper is visible.
             */
            if (visible == true) {
                handler.post(trainer);
            } else {
                handler.removeCallbacks(trainer);
            }
        }

        /**
         * {@inheritDoc}
         */
        @Override
        public void onSurfaceDestroyed(SurfaceHolder holder) {
            super.onSurfaceDestroyed(holder);
            ProgressReportingWallpaperService.visible = false;
            handler.removeCallbacks(trainer);
        }

        /**
         * {@inheritDoc}
         */
        @Override
        public void onSurfaceChanged(SurfaceHolder holder,
                                     int format, int width, int height) {
            super.onSurfaceChanged(holder, format, width, height);

            screenWidth = width;
            screenHeight = height;

            SharedPreferences preferences = PreferenceManager
                    .getDefaultSharedPreferences(
                            ProgressReportingWallpaperService.this);

            int panelsSideSize = Integer.parseInt(
                    preferences.getString("sizing", "100"));

            switch (preferences.getString("positioning", "0 0")) {
                case "lt":
                    panels[0].left = GAP_BETWEEN_PANELS;
                    panels[0].top = GAP_BETWEEN_PANELS;
                    panels[0].right = GAP_BETWEEN_PANELS + panelsSideSize;
                    panels[0].bottom = panelsSideSize + GAP_BETWEEN_PANELS;

                    panels[1].left = GAP_BETWEEN_PANELS;
                    panels[1].top = 2 * GAP_BETWEEN_PANELS + panelsSideSize;
                    panels[1].right = GAP_BETWEEN_PANELS + panelsSideSize;
                    panels[1].bottom = 2 * GAP_BETWEEN_PANELS + 2 * panelsSideSize;

                    panels[2].left = GAP_BETWEEN_PANELS;
                    panels[2].top = 3 * GAP_BETWEEN_PANELS + 2 * panelsSideSize;
                    panels[2].right = GAP_BETWEEN_PANELS + panelsSideSize;
                    panels[2].bottom = 3 * GAP_BETWEEN_PANELS + 3 * panelsSideSize;
                    break;
                case "ct":
                    panels[0].left = width / 2 - panelsSideSize / 2;
                    panels[0].top = GAP_BETWEEN_PANELS;
                    panels[0].right = width / 2 + panelsSideSize / 2;
                    panels[0].bottom = panelsSideSize + GAP_BETWEEN_PANELS;

                    panels[1].left = width / 2 - panelsSideSize / 2;
                    panels[1].top = 2 * GAP_BETWEEN_PANELS + panelsSideSize;
                    panels[1].right = width / 2 + panelsSideSize / 2;
                    panels[1].bottom = 2 * GAP_BETWEEN_PANELS + 2 * panelsSideSize;

                    panels[2].left = width / 2 - panelsSideSize / 2;
                    panels[2].top = 3 * GAP_BETWEEN_PANELS + 2 * panelsSideSize;
                    panels[2].right = width / 2 + panelsSideSize / 2;
                    panels[2].bottom = 3 * GAP_BETWEEN_PANELS + 3 * panelsSideSize;
                    break;
                case "rt":
                    panels[0].left = width - panelsSideSize - GAP_BETWEEN_PANELS;
                    panels[0].top = GAP_BETWEEN_PANELS;
                    panels[0].right = width - GAP_BETWEEN_PANELS;
                    panels[0].bottom = panelsSideSize + GAP_BETWEEN_PANELS;

                    panels[1].left = width - panelsSideSize - GAP_BETWEEN_PANELS;
                    panels[1].top = 2 * GAP_BETWEEN_PANELS + panelsSideSize;
                    panels[1].right = width - GAP_BETWEEN_PANELS;
                    panels[1].bottom = 2 * GAP_BETWEEN_PANELS + 2 * panelsSideSize;

                    panels[2].left = width - panelsSideSize - GAP_BETWEEN_PANELS;
                    panels[2].top = 3 * GAP_BETWEEN_PANELS + 2 * panelsSideSize;
                    panels[2].right = width - GAP_BETWEEN_PANELS;
                    panels[2].bottom = 3 * GAP_BETWEEN_PANELS + 3 * panelsSideSize;
                    break;
                case "lc":
                    panels[0].left = GAP_BETWEEN_PANELS;
                    panels[0].top = height / 2 - panelsSideSize / 2 -
                            GAP_BETWEEN_PANELS - panelsSideSize;
                    panels[0].right = GAP_BETWEEN_PANELS + panelsSideSize;
                    panels[0].bottom = height / 2 - panelsSideSize / 2 -
                            GAP_BETWEEN_PANELS;

                    panels[1].left = GAP_BETWEEN_PANELS;
                    panels[1].top = height / 2 - panelsSideSize / 2;
                    panels[1].right = GAP_BETWEEN_PANELS + panelsSideSize;
                    panels[1].bottom = height / 2 + panelsSideSize / 2;

                    panels[2].left = GAP_BETWEEN_PANELS;
                    panels[2].top = height / 2 + panelsSideSize / 2 +
                            GAP_BETWEEN_PANELS;
                    panels[2].right = GAP_BETWEEN_PANELS + panelsSideSize;
                    panels[2].bottom = height / 2 + panelsSideSize / 2 +
                            GAP_BETWEEN_PANELS + panelsSideSize;
                    break;
                case "cc":
                    panels[0].left = width / 2 - panelsSideSize / 2;
                    panels[0].top = height / 2 - panelsSideSize / 2 -
                            GAP_BETWEEN_PANELS - panelsSideSize;
                    panels[0].right = width / 2 + panelsSideSize / 2;
                    panels[0].bottom = height / 2 - panelsSideSize / 2 -
                            GAP_BETWEEN_PANELS;

                    panels[1].left = width / 2 - panelsSideSize / 2;
                    panels[1].top = height / 2 - panelsSideSize / 2;
                    panels[1].right = width / 2 + panelsSideSize / 2;
                    panels[1].bottom = height / 2 + panelsSideSize / 2;

                    panels[2].left = width / 2 - panelsSideSize / 2;
                    panels[2].top = height / 2 + panelsSideSize / 2 +
                            GAP_BETWEEN_PANELS;
                    panels[2].right = width / 2 + panelsSideSize / 2;
                    panels[2].bottom = height / 2 + panelsSideSize / 2 +
                            GAP_BETWEEN_PANELS + panelsSideSize;
                    break;
                case "rc":
                    panels[0].left = width - panelsSideSize - GAP_BETWEEN_PANELS;
                    panels[0].top = height / 2 - panelsSideSize / 2 -
                            GAP_BETWEEN_PANELS - panelsSideSize;
                    panels[0].right = width - GAP_BETWEEN_PANELS;
                    panels[0].bottom = height / 2 - panelsSideSize / 2 -
                            GAP_BETWEEN_PANELS;

                    panels[1].left = width - panelsSideSize - GAP_BETWEEN_PANELS;
                    panels[1].top = height / 2 - panelsSideSize / 2;
                    panels[1].right = width - GAP_BETWEEN_PANELS;
                    panels[1].bottom = height / 2 + panelsSideSize / 2;

                    panels[2].left = width - panelsSideSize - GAP_BETWEEN_PANELS;
                    panels[2].top = height / 2 + panelsSideSize / 2 +
                            GAP_BETWEEN_PANELS;
                    panels[2].right = width - GAP_BETWEEN_PANELS;
                    panels[2].bottom = height / 2 + panelsSideSize / 2 +
                            GAP_BETWEEN_PANELS + panelsSideSize;
                    break;
                case "lb":
                    panels[0].left = GAP_BETWEEN_PANELS;
                    panels[0].top = height - 3 * GAP_BETWEEN_PANELS - 3 *
                            panelsSideSize;
                    panels[0].right = GAP_BETWEEN_PANELS + panelsSideSize;
                    panels[0].bottom = height - 3 * GAP_BETWEEN_PANELS - 2 *
                            panelsSideSize;

                    panels[1].left = GAP_BETWEEN_PANELS;
                    panels[1].top = height - 2 * GAP_BETWEEN_PANELS - 2 *
                            panelsSideSize;
                    panels[1].right = GAP_BETWEEN_PANELS + panelsSideSize;
                    panels[1].bottom = height - 2 * GAP_BETWEEN_PANELS -
                            panelsSideSize;

                    panels[2].left = GAP_BETWEEN_PANELS;
                    panels[2].top = height - GAP_BETWEEN_PANELS - panelsSideSize;
                    panels[2].right = GAP_BETWEEN_PANELS + panelsSideSize;
                    panels[2].bottom = height - GAP_BETWEEN_PANELS;
                    break;
                case "cb":
                    panels[0].left = width / 2 - panelsSideSize / 2;
                    panels[0].top = height - 3 * GAP_BETWEEN_PANELS - 3 *
                            panelsSideSize;
                    panels[0].right = width / 2 + panelsSideSize / 2;
                    panels[0].bottom = height - 3 * GAP_BETWEEN_PANELS - 2 *
                            panelsSideSize;

                    panels[1].left = width / 2 - panelsSideSize / 2;
                    panels[1].top = height - 2 * GAP_BETWEEN_PANELS - 2 *
                            panelsSideSize;
                    panels[1].right = width / 2 + panelsSideSize / 2;
                    panels[1].bottom = height - 2 * GAP_BETWEEN_PANELS -
                            panelsSideSize;

                    panels[2].left = width / 2 - panelsSideSize / 2;
                    panels[2].top = height - GAP_BETWEEN_PANELS - panelsSideSize;
                    panels[2].right = width / 2 + panelsSideSize / 2;
                    panels[2].bottom = height - GAP_BETWEEN_PANELS;
                    break;
                case "rb":
                    panels[0].left = width - panelsSideSize - GAP_BETWEEN_PANELS;
                    panels[0].top = height - 3 * GAP_BETWEEN_PANELS - 3 *
                            panelsSideSize;
                    panels[0].right = width - GAP_BETWEEN_PANELS;
                    panels[0].bottom = height - 3 * GAP_BETWEEN_PANELS - 2 *
                            panelsSideSize;

                    panels[1].left = width - panelsSideSize - GAP_BETWEEN_PANELS;
                    panels[1].top = height - 2 * GAP_BETWEEN_PANELS - 2 *
                            panelsSideSize;
                    panels[1].right = width - GAP_BETWEEN_PANELS;
                    panels[1].bottom = height - 2 * GAP_BETWEEN_PANELS -
                            panelsSideSize;

                    panels[2].left = width - panelsSideSize - GAP_BETWEEN_PANELS;
                    panels[2].top = height - GAP_BETWEEN_PANELS - panelsSideSize;
                    panels[2].right = width - GAP_BETWEEN_PANELS;
                    panels[2].bottom = height - GAP_BETWEEN_PANELS;
                    break;
                default:
                    break;
            }

            delay = Long.parseLong(preferences.getString("loading",
                    "" + DEFAULT_DELAY));
        }
    }
}
\end{verbatim}

\textbf{\underline{VotingWidget.java}}
\begin{verbatim}
package eu.veldsoft.vitosha.trade;

import android.appwidget.AppWidgetManager;
import android.appwidget.AppWidgetProvider;
import android.content.Context;
import android.widget.RemoteViews;

import eu.veldsoft.vitosha.trade.dummy.InputData;

/**
 * User voting widget.
 */
public class VotingWidget extends AppWidgetProvider {

    /**
     *
     * @param context
     * @param appWidgetManager
     * @param appWidgetId
     */
    static void updateAppWidget(Context context, AppWidgetManager appWidgetManager,
                                int appWidgetId) {

        CharSequence widgetText = VotingWidgetConfigureActivity.loadTitlePref(context, appWidgetId);

        RemoteViews views = new RemoteViews(context.getPackageName(), R.layout.voting_widget);
        views.setTextViewText(R.id.appwidget_text, widgetText);

        //TODO Information should be taken from other source.
        views.setTextViewText(R.id.symbol_ticker, InputData.SYMBOL);
        views.setTextViewText(R.id.current_value, ""+InputData.OPEN[InputData.OPEN.length-1]);

        appWidgetManager.updateAppWidget(appWidgetId, views);
    }

    /**
     * {@inheritDoc}
     */
    @Override
    public void onUpdate(Context context, AppWidgetManager appWidgetManager, int[] appWidgetIds) {
        for (int appWidgetId : appWidgetIds) {
            updateAppWidget(context, appWidgetManager, appWidgetId);
        }
    }

    /**
     * {@inheritDoc}
     */
    @Override
    public void onDeleted(Context context, int[] appWidgetIds) {
        for (int appWidgetId : appWidgetIds) {
            VotingWidgetConfigureActivity.deleteTitlePref(context, appWidgetId);
        }
    }

    /**
     * {@inheritDoc}
     */
    @Override
    public void onEnabled(Context context) {
    }

    /**
     * {@inheritDoc}
     */
    @Override
    public void onDisabled(Context context) {
    }
}
\end{verbatim}

\textbf{\underline{VotingWidgetConfigureActivity.java}}
\begin{verbatim}
package eu.veldsoft.vitosha.trade;

import android.app.Activity;
import android.appwidget.AppWidgetManager;
import android.content.Context;
import android.content.Intent;
import android.content.SharedPreferences;
import android.os.Bundle;
import android.view.View;
import android.widget.EditText;

/**
 * Voting widget preference activity.
 */
public class VotingWidgetConfigureActivity extends Activity {

    /** */
    private static final String PREFS_NAME = "eu.veldsoft.vitosha.trade.VotingWidget";

    /** */
    private static final String PREF_PREFIX_KEY = "appwidget_";

    /** */
    int mAppWidgetId = AppWidgetManager.INVALID_APPWIDGET_ID;

    /** */
    EditText mAppWidgetText;

    /** */
    View.OnClickListener mOnClickListener = new View.OnClickListener() {
        public void onClick(View v) {
            final Context context = VotingWidgetConfigureActivity.this;

            String widgetText = mAppWidgetText.getText().toString();
            saveTitlePref(context, mAppWidgetId, widgetText);

            AppWidgetManager appWidgetManager = AppWidgetManager.getInstance(context);
            VotingWidget.updateAppWidget(context, appWidgetManager, mAppWidgetId);

            Intent resultValue = new Intent();
            resultValue.putExtra(AppWidgetManager.EXTRA_APPWIDGET_ID, mAppWidgetId);
            setResult(RESULT_OK, resultValue);
            finish();
        }
    };

    /** */
    public VotingWidgetConfigureActivity() {
        super();
    }

    /**
     *
     * @param context
     * @param appWidgetId
     * @param text
     */
    static void saveTitlePref(Context context, int appWidgetId, String text) {
        SharedPreferences.Editor prefs = context.getSharedPreferences(PREFS_NAME, 0).edit();
        prefs.putString(PREF_PREFIX_KEY + appWidgetId, text);
        prefs.apply();
    }

    /**
     *
     * @param context
     * @param appWidgetId
     * @return
     */
    static String loadTitlePref(Context context, int appWidgetId) {
        SharedPreferences prefs = context.getSharedPreferences(PREFS_NAME, 0);
        String titleValue = prefs.getString(PREF_PREFIX_KEY + appWidgetId, null);
        if (titleValue != null) {
            return titleValue;
        } else {
            return "";
        }
    }

    /**
     *
     * @param context
     * @param appWidgetId
     */
    static void deleteTitlePref(Context context, int appWidgetId) {
        SharedPreferences.Editor prefs = context.getSharedPreferences(PREFS_NAME, 0).edit();
        prefs.remove(PREF_PREFIX_KEY + appWidgetId);
        prefs.apply();
    }

    /**
     * {@inheritDoc}
     */
    @Override
    public void onCreate(Bundle icicle) {
        super.onCreate(icicle);

        setResult(RESULT_CANCELED);

        setContentView(R.layout.voting_widget_configure);
        mAppWidgetText = (EditText) findViewById(R.id.appwidget_text);
        findViewById(R.id.add_button).setOnClickListener(mOnClickListener);

        Intent intent = getIntent();
        Bundle extras = intent.getExtras();
        if (extras != null) {
            mAppWidgetId = extras.getInt(
                    AppWidgetManager.EXTRA_APPWIDGET_ID, AppWidgetManager.INVALID_APPWIDGET_ID);
        }

        if (mAppWidgetId == AppWidgetManager.INVALID_APPWIDGET_ID) {
            finish();
            return;
        }

        mAppWidgetText.setText(loadTitlePref(VotingWidgetConfigureActivity.this, mAppWidgetId));
    }
}
\end{verbatim}

\textbf{\underline{ConsolePredictor.java}}
\begin{verbatim}
package eu.veldsoft.vitosha.trade;

import eu.veldsoft.vitosha.trade.dummy.InputData;
import eu.veldsoft.vitosha.trade.engine.Predictor;

/**
 * Single entry point class for command line application interface.
 */
public class ConsolePredictor {
    public static void main(String[] args) {
        Predictor predictor = new Predictor();
        predictor.initialize();
        predictor.train();
        predictor.predict();
    }
}
\end{verbatim}

\textbf{\underline{HttpHelper.java}}
\begin{verbatim}
package eu.veldsoft.vitosha.trade.communication;

import org.json.JSONArray;
import org.json.JSONException;
import org.json.JSONObject;

import java.io.IOException;
import java.io.UnsupportedEncodingException;
import java.util.ArrayList;
import java.util.List;

import cz.msebera.android.httpclient.HttpResponse;
import cz.msebera.android.httpclient.NameValuePair;
import cz.msebera.android.httpclient.client.ClientProtocolException;
import cz.msebera.android.httpclient.client.HttpClient;
import cz.msebera.android.httpclient.client.entity.UrlEncodedFormEntity;
import cz.msebera.android.httpclient.client.methods.HttpPost;
import cz.msebera.android.httpclient.impl.client.DefaultHttpClient;
import cz.msebera.android.httpclient.message.BasicNameValuePair;
import cz.msebera.android.httpclient.util.EntityUtils;
import eu.veldsoft.vitosha.trade.dummy.InputData;

/**
 * It is used for HTTP communication with the remote server.
 */
public class HttpHelper {
    /**
     * Number of ANN found as response of the request.
     */
    private static final String JSON_SIZE_KEY = "size";
    /**
     * Time series symbol ticker.
     */
    private static final String JSON_SYMBOL_KEY = "symbol";
    /**
     * Time series period as integer number of minutes.
     */
    private static final String JSON_PERIOD_KEY = "period";
    /**
     * Fitness value of the ANN.
     */
    private static final String JSON_FITNESS_KEY = "fitness";
    /**
     * Array with neurons flags.
     */
    private static final String JSON_FLAGS_KEY = "flags";
    /**
     * Array with ANN weights.
     */
    private static final String JSON_WEIGHTS_KEY = "weights";
    /**
     * Array with ANN connections activities.
     */
    private static final String JSON_ACTIVITIES_KEY = "activities";
    /**
     * Number of training examples.
     */
    private static final String JSON_NUMBER_OF_EXAMPLES_KEY = "numberOfExamples";
    /**
     * Time array.
     */
    private static final String JSON_TIME_KEY = "time";
    /**
     * Open array.
     */
    private static final String JSON_OPEN_KEY = "open";
    /**
     * Low array.
     */
    private static final String JSON_LOW_KEY = "low";
    /**
     * High array.
     */
    private static final String JSON_HIGH_KEY = "high";
    /**
     * Close array.
     */
    private static final String JSON_CLOSE_KEY = "close";
    /**
     * Volume array.
     */
    private static final String JSON_VOLUME_KEY = "volume";
    /**
     * Load random ANN remote server script name.
     */
    private final String LOAD_RANDOM_ANN_SCRIPT = "logic/json_load_random_ann.php";
    /**
     * Loading training set for particular ticker and time period.
     */
    private final String LOAD_TRAINING_SET_SCRIPT = "logic/json_load_training_set.php";
    /**
     * Report retrained ANN remote server script name.
     */
    private final String SAVE_RETRAINED_ANN_SCRIPT = "logic/save_retrained_ann.php";
    /**
     * Number of neurons for the ANN.
     */
    private final String JSON_NUMBER_OF_NEURONS_KEY = "numberOfNeurons";
    /**
     * Remote server URl address.
     */
    private final String url;

    /**
     * Constructor with all parameters needed.
     *
     * @param url Remote server URL address.
     */
    public HttpHelper(String url) {
        this.url = url;
    }

    /*
     * Load remote data into input data structure.
     *
     * @return True if the loading was successful, false otherwise.
     */
    public boolean load() {
        String symbol = InputData.SYMBOL;
        int period = InputData.PERIOD;
        int[] flags = InputData.NEURONS;
        double[][] weights = InputData.WEIGHTS;
        double[][] activities = InputData.ACTIVITIES;
        long[] time = InputData.TIME;
        double[] open = InputData.OPEN;
        double[] low = InputData.LOW;
        double[] high = InputData.HIGH;
        double[] close = InputData.HIGH;
        double[] volume = InputData.VOLUME;

        HttpClient client = new DefaultHttpClient();
        client.getParams().setParameter("http.protocol.content-charset", "UTF-8");

        /*
         * Load randomly selected ANN.
         */
        HttpPost post = new HttpPost("http://" + url.trim() + "/" + LOAD_RANDOM_ANN_SCRIPT);

        try {
            HttpResponse response = client.execute(post);

            JSONObject result = new JSONObject(EntityUtils.toString(response.getEntity(), "UTF-8"));

            int size = result.getInt(JSON_SIZE_KEY);

            /*
             * If there is no ANN on the server side nothing can be loaded.
             */
            if (size <= 0) {
                return false;
            }

            /*
             * Extract JSON from HTTP response.
             */
            symbol = result.getString(JSON_SYMBOL_KEY);

            period = result.getInt(JSON_PERIOD_KEY);

            double fitness = result.getDouble(JSON_FITNESS_KEY);

            int numberOfNeurons = result.getInt(JSON_NUMBER_OF_NEURONS_KEY);

            flags = new int[numberOfNeurons];
            JSONArray array1 = result.getJSONArray(JSON_FLAGS_KEY);
            for (int i = 0; i < array1.length(); i++) {
                flags[i] = array1.getInt(i);
            }

            //TODO Matrix transpose is possible.
            weights = new double[numberOfNeurons][numberOfNeurons];
            JSONArray array2 = result.getJSONArray(JSON_WEIGHTS_KEY);
            for (int j = 0; j < array2.length(); j++) {
                JSONArray array3 = array2.getJSONArray(j);
                for (int i = 0; i < array3.length(); i++) {
                    weights[i][j] = array3.getDouble(i);
                }
            }

            //TODO Matrix transpose is possible.
            activities = new double[numberOfNeurons][numberOfNeurons];
            JSONArray array4 = result.getJSONArray(JSON_ACTIVITIES_KEY);
            for (int j = 0; j < array4.length(); j++) {
                JSONArray array5 = array4.getJSONArray(j);
                for (int i = 0; i < array5.length(); i++) {
                    activities[i][j] = array5.getDouble(i);
                }
            }
        } catch (ClientProtocolException exception) {
            return false;
        } catch (IOException exception) {
            return false;
        } catch (JSONException exception) {
            return false;
        } catch (Exception exception) {
            return false;
        }

        /*
         * Load training set for the selected ANN.
         */
        post = new HttpPost("http://" + url.trim() + "/" + LOAD_TRAINING_SET_SCRIPT);
        List<NameValuePair> pairs = new ArrayList<NameValuePair>();
        pairs.add(new BasicNameValuePair("symbol", symbol));
        pairs.add(new BasicNameValuePair("period", "" + period));
        try {
            post.setEntity(new UrlEncodedFormEntity(pairs));
        } catch (UnsupportedEncodingException e) {
            return false;
        }

        try {
            HttpResponse response = client.execute(post);

            JSONObject result = new JSONObject(EntityUtils.toString(response.getEntity(), "UTF-8"));

            int size = result.getInt(JSON_NUMBER_OF_EXAMPLES_KEY);

            /*
             * If there is no ANN training set on the server side nothing can be loaded.
             */
            if (size <= 0) {
                return false;
            }

            time = new long[size];
            JSONArray array1 = result.getJSONArray(JSON_TIME_KEY);
            for (int i = 0; i < array1.length(); i++) {
                time[i] = array1.getLong(i);
            }

            open = new double[size];
            JSONArray array2 = result.getJSONArray(JSON_OPEN_KEY);
            for (int i = 0; i < array2.length(); i++) {
                open[i] = array2.getDouble(i);
            }

            low = new double[size];
            JSONArray array3 = result.getJSONArray(JSON_LOW_KEY);
            for (int i = 0; i < array3.length(); i++) {
                low[i] = array3.getDouble(i);
            }

            high = new double[size];
            JSONArray array4 = result.getJSONArray(JSON_HIGH_KEY);
            for (int i = 0; i < array4.length(); i++) {
                high[i] = array4.getDouble(i);
            }

            close = new double[size];
            JSONArray array5 = result.getJSONArray(JSON_CLOSE_KEY);
            for (int i = 0; i < array5.length(); i++) {
                close[i] = array5.getDouble(i);
            }

            volume = new double[size];
            JSONArray array6 = result.getJSONArray(JSON_VOLUME_KEY);
            for (int i = 0; i < array6.length(); i++) {
                volume[i] = array6.getDouble(i);
            }
        } catch (ClientProtocolException exception) {
            return false;
        } catch (IOException exception) {
            return false;
        } catch (JSONException exception) {
            return false;
        } catch (Exception exception) {
            return false;
        }

        /*
         * Load data in the global data structure.
         */
        InputData.SYMBOL = symbol;
        InputData.PERIOD = period;
        InputData.NEURONS = flags;
        InputData.WEIGHTS = weights;
        InputData.ACTIVITIES = activities;
        InputData.TIME = time;
        InputData.OPEN = open;
        InputData.LOW = low;
        InputData.HIGH = high;
        InputData.CLOSE = close;
        InputData.VOLUME = volume;
        InputData.RATES = new double[][]{InputData.OPEN, InputData.LOW, InputData.HIGH, InputData.CLOSE};

        return true;
    }

    /**
     * Store calculated ANN weights on the remote web server.
     *
     * @return True if the saving was successful, false otherwise.
     */
    public boolean store() {
        HttpClient client = new DefaultHttpClient();
        client.getParams().setParameter("http.protocol.content-charset", "UTF-8");

        /*
         * Store retrained ANN.
         */
        HttpPost post = new HttpPost("http://" + url.trim() + "/" + SAVE_RETRAINED_ANN_SCRIPT);
        List<NameValuePair> pairs = new ArrayList<NameValuePair>();

        pairs.add(new BasicNameValuePair("symbol", InputData.SYMBOL));
        pairs.add(new BasicNameValuePair("period", "" + InputData.PERIOD));
        pairs.add(new BasicNameValuePair("fitness", "" + InputData.FITNESS));
        pairs.add(new BasicNameValuePair("number_of_neurons", "" + InputData.NEURONS.length));

        String flags = "";
        for (int i = 0; i < InputData.NEURONS.length; i++) {
            flags += InputData.NEURONS[i] + " ";
        }
        pairs.add(new BasicNameValuePair("flags", "" + flags.trim()));

        //TODO Matrix transpose is possible.
        String weights = "";
        for (int j = 0; j < InputData.NEURONS.length; j++) {
            for (int i = 0; i < InputData.NEURONS.length; i++) {
                weights += InputData.WEIGHTS[i][j] + " ";
            }
            weights = weights.trim() + "\r\n";
        }
        pairs.add(new BasicNameValuePair("weights", "" + weights.trim()));

        //TODO Matrix transpose is possible.
        String activites = "";
        for (int j = 0; j < InputData.NEURONS.length; j++) {
            for (int i = 0; i < InputData.NEURONS.length; i++) {
                activites += InputData.ACTIVITIES[i][j] + " ";
            }
            activites = activites.trim() + "\r\n";
        }
        pairs.add(new BasicNameValuePair("activites", "" + activites.trim()));

        try {
            post.setEntity(new UrlEncodedFormEntity(pairs));
            client.execute(post);
        } catch (ClientProtocolException exception) {
            return false;
        } catch (UnsupportedEncodingException exception) {
            return false;
        } catch (IOException exception) {
            return false;
        }

        return true;
    }
}
\end{verbatim}

\textbf{\underline{NeuronType.java}}
\begin{verbatim}
package eu.veldsoft.vitosha.trade.communication;

/**
 * Numeric constants for neurons type.
 */
public enum NeuronType {

    /**
     * Regular neuron flag.
     */
    REGULAR(0x00),

    /**
     * Bias neuron flag.
     */
    BIAS(0x01),

    /**
     * Input neuron flag.
     */
    INPUT(0x02),

    /**
     * Input and bias neuron flag.
     */
    INPUT_BIAS(0x03),

    /**
     * Output neuron flag.
     */
    OUTPUT(0x04),

    /**
     * Output and bias neuron flag.
     */
    OUTPUT_BIAS(0x05),

    /**
     * Output and input neuron flag.
     */
    OUTPUT_INPUT(0x06),

    /**
     * Output, input and bias neuron flag.
     */
    OUTPUT_INPUT_BIAS(0x07);

    /*
     * Numeric value representation.
     */
    private final int value;

    /**
     * Constructor with all parameters.
     *
     * @param value
     */
    private NeuronType(int value) {
        this.value = value;
    }

    /**
     * Value factory function.
     *
     * @param type Numerical type representation.
     * @return Corresponding enumeration or regular if there is no correspondence.
     */
    public static NeuronType valueOf(int type) {
        for (NeuronType item : NeuronType.values()) {
            if (item.value() == type) {
                return item;
            }
        }

        return REGULAR;
    }

    /**
     * Value getter.
     *
     * @return Numeric representation of the type.
     */
    public int value() {
        return value;
    }
}
\end{verbatim}

\textbf{\underline{TimePeriod.java}}
\begin{verbatim}
package eu.veldsoft.vitosha.trade.communication;

/**
 * Time series fixed time periods.
 */
public enum TimePeriod {

    /**
     * No time period at all.
     */
    NONE(0, ""),

    /**
     * One minute.
     */
    M1(1, "M1"),

    /**
     * Five minutes.
     */
    M5(5, "M5"),

    /**
     * Fifteen minutes.
     */
    M15(15, "M15"),

    /**
     * Thirty minutes.
     */
    M30(30, "M30"),

    /**
     * One hour.
     */
    H1(60, "H1"),

    /**
     * Four hours.
     */
    H4(240, "H4"),

    /**
     * One day.
     */
    D1(1440, "D1"),

    /**
     * One week.
     */
    W1(10080, "W1"),

    /**
     * One month.
     */
    MN1(43200, "MN1");

    /**
     * Time period as number of minutes.
     */
    private int minutes;

    /**
     * Time period as text description.
     */
    private String name;

    /**
     * Constructor with all parameters.
     *
     * @param minutes Minutes as numbers.
     * @param name    Interval as name.
     */
    private TimePeriod(int minutes, String name) {
        this.minutes = minutes;
        this.name = name;
    }

    /**
     * Factory function for object reference from time interval.
     *
     * @param minutes Time interval in minutes.
     * @return Time period as object.
     * @throws RuntimeException Rise exception if there is no such time interval in minutes.
     */
    public static TimePeriod value(int minutes) throws RuntimeException {
        for (TimePeriod item : TimePeriod.values()) {
            if (item.minutes == minutes) {
                return item;
            }
        }

        // TODO Report exception.

        return NONE;
    }

    /**
     * Factory function for object reference from form interval name.
     *
     * @param name Time period name.
     * @return Time period as object.
     * @throws RuntimeException Rise exception if there is no such time interval in minutes.
     */
    public static TimePeriod value(String name) throws RuntimeException {
        for (TimePeriod item : TimePeriod.values()) {
            if (name.equals(item.name) == true) {
                return item;
            }
        }

        // TODO Report exception.

        return NONE;
    }

    /**
     * {@inheritDoc}
     */
    @Override
    public String toString() {
        return name;
    }
}
\end{verbatim}

\textbf{\underline{ApacheOptimizer.java}}
\begin{verbatim}
package eu.veldsoft.vitosha.trade.engine;

import org.apache.commons.math3.genetics.Chromosome;
import org.apache.commons.math3.genetics.ElitisticListPopulation;
import org.apache.commons.math3.genetics.FixedElapsedTime;
import org.apache.commons.math3.genetics.GeneticAlgorithm;
import org.apache.commons.math3.genetics.Population;
import org.apache.commons.math3.genetics.TournamentSelection;
import org.apache.commons.math3.genetics.UniformBinaryMutation;
import org.apache.commons.math3.genetics.UniformCrossover;
import org.apache.commons.math3.genetics.WeightsChromosome;
import org.encog.neural.networks.BasicNetwork;
import org.encog.neural.networks.training.propagation.Propagation;

import java.util.LinkedList;
import java.util.List;

/**
 * Apache Common Math Genetic Algorithms based optimizer.
 */
public class ApacheOptimizer implements Optimizer {
    /**
     *
     */
    private final long optimizationTimeout;

    /**
     *
     */
    private final BasicNetwork network;

    /**
     *
     */
    private final Propagation train;

    /**
     *
     */
    private final int populationSize;

    /**
     *
     */
    private final int tournamentArity;

    /**
     *
     */
    private final double crossoverRate;

    /**
     *
     */
    private final double mutationRate;

    /**
     *
     */
    private final double elitismRate;

    /**
     * @param optimizationTimeout
     * @param network
     * @param train
     * @param populationSize
     * @param tournamentArity
     * @param crossoverRate
     * @param mutationRate
     * @param elitismRate
     */
    public ApacheOptimizer(long optimizationTimeout, BasicNetwork network, Propagation train, int populationSize, int tournamentArity, double crossoverRate, double mutationRate, double elitismRate) {
        this.optimizationTimeout = optimizationTimeout;
        this.network = network;
        this.train = train;
        this.populationSize = populationSize;
        this.tournamentArity = tournamentArity;
        this.crossoverRate = crossoverRate;
        this.mutationRate = mutationRate;
        this.elitismRate = elitismRate;
    }

    /**
     * {@inheritDoc}
     */
    @Override
    public List<Double> optimize(List<Double> weights) {

        /*
         * Generate population.
         */
        List<Chromosome> chromosomes = new LinkedList<Chromosome>();
        for (int i = 0; i < populationSize; i++) {
            chromosomes.add(new WeightsChromosome(weights, true, network, train));
        }
        Population initial = new ElitisticListPopulation(chromosomes,
                2 * chromosomes.size(), elitismRate);

        /*
         * Initialize genetic algorithm.
         */
        GeneticAlgorithm algorithm = new GeneticAlgorithm(
                new UniformCrossover<WeightsChromosome>(0.5),
                crossoverRate, new UniformBinaryMutation(),
                mutationRate, new TournamentSelection(tournamentArity));

        /*
         * Run optimization.
         */
        Population optimized = algorithm.evolve(initial,
                new FixedElapsedTime(optimizationTimeout));

        /*
         * Obtain result.
         */
        weights = ((WeightsChromosome)
                optimized.getFittestChromosome()).
                getRepresentation();

        return weights;
    }
}
\end{verbatim}

\textbf{\underline{MoeaOptimizer.java}}
\begin{verbatim}
package eu.veldsoft.vitosha.trade.engine;

import org.encog.neural.networks.BasicNetwork;
import org.encog.neural.networks.training.propagation.Propagation;
import org.moeaframework.algorithm.AbstractAlgorithm;
import org.moeaframework.algorithm.single.AggregateObjectiveComparator;
import org.moeaframework.algorithm.single.DifferentialEvolution;
import org.moeaframework.algorithm.single.LinearDominanceComparator;
import org.moeaframework.core.Initialization;
import org.moeaframework.core.NondominatedPopulation;
import org.moeaframework.core.Problem;
import org.moeaframework.core.Solution;
import org.moeaframework.core.operator.InjectedInitialization;
import org.moeaframework.core.operator.real.DifferentialEvolutionSelection;
import org.moeaframework.core.operator.real.DifferentialEvolutionVariation;
import org.moeaframework.core.variable.EncodingUtils;
import org.moeaframework.problem.misc.AnnErrorMinimizationProblem;

import java.util.ArrayList;
import java.util.List;

/**
 * MOEA Framework based optimizer.
 */
public class MoeaOptimizer implements Optimizer {
    /**
     *
     */
    private final long optimizationTimeout;

    /**
     *
     */
    private final BasicNetwork network;

    /**
     *
     */
    private final Propagation train;

    /**
     *
     */
    private final int populationSize;

    /**
     *
     */
    private final double crossoverRate;

    /**
     *
     */
    private final double scalingFactor;

    /**
     *
     * @param optimizationTimeout
     * @param network
     * @param train
     * @param populationSize
     * @param crossoverRate
     * @param scalingFactor
     */
    public MoeaOptimizer(long optimizationTimeout, BasicNetwork network, Propagation train, int populationSize, double crossoverRate, double scalingFactor) {
        this.optimizationTimeout = optimizationTimeout;
        this.network = network;
        this.train = train;
        this.populationSize = populationSize;
        this.crossoverRate = crossoverRate;
        this.scalingFactor = scalingFactor;
    }

    /**
     * {@inheritDoc}
     */
    @Override
    public List<Double> optimize(List<Double> weights) {
        Problem problem = new AnnErrorMinimizationProblem(weights, network, train);

        List<Solution> solutions = new ArrayList<Solution>();
        for (int i = 0; i < populationSize; i++) {
            Solution solution = problem.newSolution();
            solutions.add(solution);
        }

        AggregateObjectiveComparator comparator = new LinearDominanceComparator();
        Initialization initialization = new InjectedInitialization(problem, populationSize, solutions);
        DifferentialEvolutionSelection selection = new DifferentialEvolutionSelection();
        DifferentialEvolutionVariation variation = new DifferentialEvolutionVariation(crossoverRate, scalingFactor);

        AbstractAlgorithm algorithm = new DifferentialEvolution(problem, comparator, initialization, selection, variation);

        long stop = System.currentTimeMillis() + optimizationTimeout;
        while (System.currentTimeMillis() < stop) {
            algorithm.step();
        }

        weights = new ArrayList<Double>();
        NondominatedPopulation population = algorithm.getResult();
        if (population.size() > 0) {
            for (Double value : EncodingUtils.getReal(population.get(0))) {
                weights.add(value);
            }
        }

        return weights;
    }
}
\end{verbatim}

\textbf{\underline{Optimizer.java}}
\begin{verbatim}
package eu.veldsoft.vitosha.trade.engine;

import java.util.List;

/**
 * Optimizer interface.
 */
public interface Optimizer {
    /**
     * Single cycle of optimization.
     *
     * @param weights Initial weights.
     * @return Weights after single cycle of optimization.
     */
    public List<Double> optimize(List<Double> weights);
}
\end{verbatim}

\textbf{\underline{Predictor.java}}
\begin{verbatim}
package eu.veldsoft.vitosha.trade.engine;

import org.encog.engine.network.activation.ActivationFunction;
import org.encog.engine.network.activation.ActivationTANH;
import org.encog.ml.data.MLData;
import org.encog.ml.data.MLDataSet;
import org.encog.ml.data.basic.BasicMLData;
import org.encog.ml.data.basic.BasicMLDataSet;
import org.encog.ml.data.market.MarketDataDescription;
import org.encog.ml.data.market.MarketDataType;
import org.encog.ml.data.market.MarketMLDataSet;
import org.encog.ml.data.market.TickerSymbol;
import org.encog.ml.data.market.loader.LoadedMarketData;
import org.encog.ml.data.market.loader.MarketLoader;
import org.encog.neural.networks.BasicNetwork;
import org.encog.neural.networks.layers.BasicLayer;
import org.encog.neural.networks.training.propagation.Propagation;
import org.encog.neural.networks.training.propagation.resilient.ResilientPropagation;

import java.util.ArrayList;
import java.util.Arrays;
import java.util.Collection;
import java.util.Date;
import java.util.HashMap;
import java.util.List;
import java.util.Map;
import java.util.Random;
import java.util.Set;

import eu.veldsoft.vitosha.trade.communication.NeuronType;
import eu.veldsoft.vitosha.trade.dummy.InputData;

/**
 * Forecasting engine main class.
 */
public class Predictor {

    /**
     * Pseudo-random number generator.
     */
    private static final Random PRNG = new Random();

    // TODO Put all colors in the settings dialog.

    /**
     * Colors used in the charts.
     */
    private static final int CHART_COLORS[] = {
            (95 << 24 | 0 << 16 | 255 << 8 | 0),
            (95 << 24 | 255 << 16 | 0 << 8 | 0),
    };

    /**
     * Colors used to visualize neural networks.
     */
    private static final int ANN_COLORS[] = {
            (95 << 24 | 0 << 16 | 255 << 8 | 0),
            (95 << 24 | 255 << 16 | 255 << 8 | 255),
            (95 << 24 | 0 << 16 | 0 << 8 | 255),
            (95 << 24 | 255 << 16 | 255 << 8 | 255),
            (95 << 24 | 255 << 16 | 0 << 8 | 0),
    };

    /**
     * Neural network object.
     */
    private static BasicNetwork network = new BasicNetwork();

    /**
     * Training examples data set.
     */
    private static MLDataSet examples = null;

    /**
     * Data of the forecast.
     */
    private static MLData forecast = null;

    /**
     * Calculated output data.
     */
    private static MLData output = null;

    /**
     * Training rule object.
     */
    private static Propagation train = null;

    /**
     * Lowest and highest values of particular activation function. It is used for time series scaling.
     *
     * @param activation Activation function object.
     * @return Array with two values - lowest in the first index and highest in the second index.
     */
    private static double[] findLowAndHigh(ActivationFunction activation) {
        /*
         * Use range of double values.
         */
        double check[] = {
                Double.MIN_VALUE, -0.000001, -0.00001, -0.0001,
                -0.001, -0.01, -0.1, -1, -10, -100, -1000,
                -10000, -100000, -1000000, 0, 0.000001, 0.00001,
                0.0001, 0.001, 0.01, 0.1, 1, 10, 100, 1000, 10000,
                100000, 1000000, Double.MAX_VALUE};

        /*
         * Map the range of double values to activation function output.
         */
        activation.activationFunction(check, 0, check.length);

        /*
         * Soft the result of the activation fuction output.
         */
        Arrays.sort(check);

        /*
         * Return minimum and maximum values of the activation function output.
         */
        return new double[]{check[0], check[check.length - 1]};
    }

    /**
     * Initialize common class members.
     */
    public void initialize() {
        Map<NeuronType, Integer> counters = new HashMap<NeuronType, Integer>();
        counters.put(NeuronType.REGULAR, 0);
        counters.put(NeuronType.BIAS, 0);
        counters.put(NeuronType.INPUT, 0);
        counters.put(NeuronType.OUTPUT, 0);

        for (int type : InputData.NEURONS) {
            counters.put(NeuronType.valueOf(type),
                    counters.get(NeuronType.valueOf(type)) + 1);
        }

        int inputSize = counters.get(NeuronType.INPUT);
        int hiddenSize = counters.get(NeuronType.REGULAR);
        int outputSize = counters.get(NeuronType.OUTPUT);

        /*
         * Network construction.
         */
        network.addLayer(new BasicLayer(null,
                true, inputSize));
        network.addLayer(new BasicLayer(new ActivationTANH(),
                true, hiddenSize));
        network.addLayer(new BasicLayer(new ActivationTANH(),
                false, outputSize));
        network.getStructure().finalizeStructure();
        network.reset();

        // TODO Load weights to the network.

        double values[] = InputData.RATES[PRNG.nextInt(InputData.RATES.length)];

        /*
         * Data construction.
         */
        MarketMLDataSet data = new MarketMLDataSet(new MarketLoader() {
            @Override
            public Collection<LoadedMarketData> load(TickerSymbol symbol,
                                                     Set<MarketDataType> types, Date start,
                                                     Date end) {
                Collection<LoadedMarketData> result = new
                        ArrayList<LoadedMarketData>();

                for (int i = 0; i < InputData.TIME.length; i++) {
                    /*
                     * Data outside of the desired time frame are not loaded.
                     */
                    if (InputData.TIME[i] < start.getTime() || end.getTime() < InputData.TIME[i]) {
                        continue;
                    }

                    LoadedMarketData value = new LoadedMarketData(new
                            Date(InputData.TIME[i]), symbol);

                    value.setData(MarketDataType.CLOSE, InputData.CLOSE[i]);
                    value.setData(MarketDataType.HIGH, InputData.HIGH[i]);
                    value.setData(MarketDataType.LOW, InputData.LOW[i]);
                    value.setData(MarketDataType.OPEN, InputData.OPEN[i]);
                    value.setData(MarketDataType.VOLUME, InputData.VOLUME[i]);
                    value.setData(MarketDataType.ADJUSTED_CLOSE, InputData.CLOSE[i]);

                    result.add(value);
                }

                return result;
            }
        }, inputSize, outputSize);

        MarketDataDescription description = new MarketDataDescription(new
                TickerSymbol(InputData.SYMBOL),
                (new MarketDataType[]{MarketDataType.CLOSE, MarketDataType.HIGH,
                        MarketDataType.LOW,
                        MarketDataType.OPEN})[(int) (Math.random() * 4)],
                true, true);
        data.addDescription(description);
        data.load(new Date(InputData.TIME[0]), new
                Date(InputData.TIME[InputData.TIME.length - 1]));
        data.generate();

        /*
         * Normalize data.
         */
        double min = values[0];
        double max = values[0];
        for (double value : values) {
            if (value < min) {
                min = value;
            }
            if (value > max) {
                max = value;
            }
        }

        /*
         * At the first index is the low value. At the second index is the high
         * value.
         *
         * There is a problem with this approach, because some activation
         * functions are zero if the argument is infinity.
         *
         * The fist layer has no activation function.
         */
        double range[] = findLowAndHigh(network.getActivation(2));

        /*
         * Prepare training set.
         */
        double input[][] = new double[values.length -
                (inputSize + outputSize)][inputSize];
        double target[][] = new double[values.length -
                (inputSize + outputSize)][outputSize];
        for (int i = 0; i < values.length - (inputSize + outputSize); i++) {
            for (int j = 0; j < inputSize; j++) {
                input[i][j] = range[0] + (range[1] - range[0]) *
                        (values[i + j] - min) / (max - min);
            }
            for (int j = 0; j < outputSize; j++) {
                target[i][j] = range[0] + (range[1] - range[0]) *
                        (values[i + inputSize + j] - min) / (max - min);
            }
        }
        examples = new BasicMLDataSet(input, target);

        /*
         * Prepare forecast set.
         */
        input = new double[1][inputSize];
        for (int j = 0; j < inputSize; j++) {
            input[0][j] = range[0] + (range[1] - range[0]) *
                    (values[values.length - inputSize + j] - min) / (max - min);
        }
        forecast = new BasicMLData(input[0]);
    }

    /**
     * Single neural network training cycle.
     */
    public void train() {
        if (network == null) {
            return;
        }
        if (examples == null) {
            return;
        }

        train = new ResilientPropagation(network, examples);

        /*
         * Switch between backpropagation and genetic algorithm.
         */
        if (PRNG.nextBoolean() == true) {
            train.iteration();
            train.finishTraining();
        } else {
            int populationSize = 4 + PRNG.nextInt(33);
            double elitismRate = PRNG.nextInt(100) / 1000D;
            double crossoverRate = PRNG.nextInt(900) / 1000D;
            double mutationRate = PRNG.nextInt(10) / 1000D;
            int tournamentArity = PRNG.nextBoolean() ? 1 : 2;
            double scalingFactor = PRNG.nextInt(500) / 1000D;
            long optimizationTimeout = 1000;

            /*
             * Obtain ANN weights.
             */
            List<Double> weights = new ArrayList<Double>();
            for (int layer = 0; layer < network.getLayerCount() - 1; layer++) {
                int bias = network.isLayerBiased(layer) ? 1 : 0;
                for (int from = 0; from < network.getLayerNeuronCount(layer) + bias; from++) {
                    for (int to = 0; to < network.getLayerNeuronCount(layer + 1); to++) {
                        weights.add(network.getWeight(layer, from, to));
                    }
                }
            }

            /* Do evolutionary optimization. */
            Optimizer optimizer = new MoeaOptimizer(optimizationTimeout, network, train, populationSize, crossoverRate, scalingFactor);
            weights = optimizer.optimize(weights);

            /*
             * Replace ANN weights.
             */
            for (int layer = 0, index = 0; layer < network.getLayerCount() - 1; layer++) {
                int bias = network.isLayerBiased(layer) ? 1 : 0;
                for (int from = 0; from < network.getLayerNeuronCount(layer) + bias; from++) {
                    for (int to = 0; to < network.getLayerNeuronCount(layer + 1); to++, index++) {
                        network.setWeight(layer, from, to, weights.get(index));
                    }
                }
            }
        }
    }

    /**
     * Neural network prediction getter.
     */
    public void predict() {
        if (network == null) {
            return;
        }
        if (forecast == null) {
            return;
        }

        output = network.compute(forecast);
    }

    /**
     * Draw forecast.
     *
     * @param pixels Array with ARGB pixels.
     * @param width  Drawing area width.
     * @param height Drawing area height.
     */
    public void drawForecast(int[] pixels, int width, int height) {
        if (network == null) {
            return;
        }
        if (forecast == null) {
            return;
        }
        if (output == null) {
            return;
        }

        /*
         * Output layer activation function is used because input layer
         * has no activation function.
         */
        double range[] = findLowAndHigh(network.getActivation(2));

        /*
         * Total number of values to be visualized.
         */
        int numberOfValues = network.getLayerNeuronCount(0) +
                network.getLayerNeuronCount(2);

        int x = 0;
        int y = 0;

        /*
         * Visualize past data.
         */
        for (int i = 0; forecast.getData() != null &&
                i < forecast.getData().length; i++) {
            int offset = (int) (height * (forecast.getData()[i] - range[0]) /
                    (range[1] - range[0]));
            for (int dx = 0; dx < width / numberOfValues; dx++) {
                for (y = height - offset; y < height; y++) {
                    pixels[x + y * width] = CHART_COLORS[0];
                }
                x++;
            }
        }

        /*
         * Visualize future data.
         */
        for (int i = 0; output.getData() != null &&
                i < output.getData().length; i++) {
            int offset = (int) (height * (output.getData()[i] - range[0]) /
                    (range[1] - range[0]));
            for (int dx = 0; dx < width / numberOfValues; dx++) {
                for (y = height - offset; y < height; y++) {
                    pixels[x + y * width] = CHART_COLORS[1];
                }
                x++;
            }
        }
    }

    /**
     * Draw ANN topology.
     *
     * @param pixels Array with ARGB pixels.
     * @param width  Drawing area width.
     * @param height Drawing area height.
     */
    public void drawAnn(int[] pixels, int width, int height) {
        if (network == null) {
            return;
        }
        if (forecast == null) {
            return;
        }
        if (output == null) {
            return;
        }

        /*
         * Artificial neural network.
         */
        double topology[][] = {
                forecast.getData(),
                new double[network.getLayerNeuronCount(0) *
                        network.getLayerNeuronCount(1)],
                new double[network.getLayerNeuronCount(1)],
                new double[network.getLayerNeuronCount(1) *
                        network.getLayerNeuronCount(2)],
                output.getData()
        };

        /*
         * At the first index is the low value. At the second index is
         * the high value.
         *
         * There is a problem with this approach, because some activation
         * functions are zero if the argument is infinity.
         *
         * The fist layer has no activation function.
         */
        double range[] = findLowAndHigh(network.getActivation(2));

        /*
         * Scale input layer data.
         */
        for (int i = 0; i < topology[0].length; i++) {
            topology[0][i] = (topology[0][i] - range[0]) /
                    (range[1] - range[0]);
        }

        /*
         * Scale output layer data.
         */
        for (int i = 0; i < topology[4].length; i++) {
            topology[4][i] = (topology[4][i] - range[0]) /
                    (range[1] - range[0]);
        }

        for (int i = 0, from = 0, to = 0, bias = network.isLayerBiased(0) ? 1 : 0; i < topology[1].length; i++) {
            if (to >= network.getLayerNeuronCount(1)) {
                to = 0;
                from++;
            }
            if ((from + bias) >= network.getLayerNeuronCount(0)) {
                from = 0;
            }
            topology[1][i] = network.getWeight(0, from, to);
            to++;
        }

        for (int i = 0, from = 0, to = 0, bias = network.isLayerBiased(1) ? 1 : 0; i < topology[3].length; i++) {
            if (to >= network.getLayerNeuronCount(2)) {
                to = 0;
                from++;
            }
            if ((from + bias) >= network.getLayerNeuronCount(1)) {
                from = 0;
            }
            topology[3][i] = network.getWeight(1, from, to);
            to++;
        }

        /*
         * Hidden layer values. Activation function of the second layer
         * is used for scaling.
         */
        range = findLowAndHigh(network.getActivation(1));
        for (int i = 0; i < topology[2].length; i++) {
            topology[2][i] = (network.getLayerOutput(1, i) - range[0]) /
                    (range[1] - range[0]);
        }

        /*
         * Weights normalization.
         */
        double min = Double.MAX_VALUE;
        double max = Double.MIN_VALUE;
        for (double value : topology[1]) {
            if (value < min) {
                min = value;
            }
            if (value > max) {
                max = value;
            }
        }
        for (double value : topology[3]) {
            if (value < min) {
                min = value;
            }
            if (value > max) {
                max = value;
            }
        }
        for (int i = 0; i < topology[1].length; i++) {
            topology[1][i] = (topology[1][i] - min) / (max - min);
        }
        for (int i = 0; i < topology[3].length; i++) {
            topology[3][i] = (topology[3][i] - min) / (max - min);
        }

        for (int x = 0, k = 0; k < ANN_COLORS.length;
             x += width / ANN_COLORS.length, k++) {
            for (int dx = 0; dx < width / ANN_COLORS.length; dx++) {
                for (int y = 0, l = 0; y < height &&
                        l < topology[k].length; y += height / topology[k].length, l++) {
                    for (int dy = 0; dy < height / topology[k].length; dy++) {
                        int alpha = (int) (((ANN_COLORS[k] >> 24) & 0xFF) * topology[k][l]);
                        int red = (int) (((ANN_COLORS[k] >> 16) & 0xFF) * topology[k][l]);
                        int green = (int) (((ANN_COLORS[k] >> 8) & 0xFF) * topology[k][l]);
                        int blue = (int) ((ANN_COLORS[k] & 0xFF) * topology[k][l]);

                        int color = alpha << 24 | red << 16 | green << 8 | blue;

                        pixels[(x + dx) + (y + dy) * width] = color;
                    }
                }
            }
        }
    }
}
\end{verbatim}

\textbf{\underline{UniformBinaryMutation.java}}
\begin{verbatim}
package org.apache.commons.math3.genetics;

import org.apache.commons.math3.exception.MathIllegalArgumentException;
import org.apache.commons.math3.exception.util.LocalizedFormats;
import org.apache.commons.math3.random.RandomGenerator;

import java.util.ArrayList;
import java.util.List;

/**
 * Uniform random bits mutation applied over vector of double values.
 */
public class UniformBinaryMutation implements MutationPolicy {

    /**
     * Pseudo random number generator.
     */
    private static final RandomGenerator PRNG =
            GeneticAlgorithm.getRandomGenerator();

    /**
     * {@inheritDoc}
     */
    @Override
    public Chromosome mutate(Chromosome original)
            throws MathIllegalArgumentException {
        if (original instanceof WeightsChromosome == false) {
            throw new MathIllegalArgumentException(
                    LocalizedFormats.INVALID_BINARY_CHROMOSOME);
        }

        /*
         * Deep copy of list values.
         */
        List<Double> representation = new ArrayList<Double>(
                ((WeightsChromosome) original).getRepresentation());

        /*
         * Mutate a single bit in each chromosome value.
         */
        for (int i = 0; i < representation.size(); i++) {
            double value = representation.get(i);
            if (PRNG.nextBoolean() == true) {
                value -= Double.MIN_VALUE;
            } else {
                value += Double.MIN_VALUE;
            }
            representation.set(i, value);
        }

        /*
         * Construct new chromosome after mutation.
         */
        return ((WeightsChromosome) original).
                newFixedLengthChromosome(representation);
    }
}
\end{verbatim}

\textbf{\underline{WeightsChromosome.java}}
\begin{verbatim}
package org.apache.commons.math3.genetics;

import org.encog.neural.networks.BasicNetwork;
import org.encog.neural.networks.training.Train;

import java.util.List;

/**
 * Chromosome with double values of the weights from the artificial neural
 * network.
 */
public class WeightsChromosome extends AbstractListChromosome<Double> {

    /**
     * Lazy initialization and buffering for the fitness value.
     */
    double fitness = -(Double.MAX_VALUE - 1);
    /**
     * Reference to external neural network object.
     */
    private BasicNetwork network = null;
    /**
     * Reference to external training strategy object.
     */
    private Train train = null;

    /**
     * Constructor used to initialize the chromosome with array of values.
     *
     * @param representation Values as array.
     * @param network        Neural network reference.
     * @param train          Neural network training strategy.
     * @throws InvalidRepresentationException Rise an exception if the values
     *                                        are not valid.
     */
    public WeightsChromosome(Double[] representation,
                             BasicNetwork network, Train train)
            throws InvalidRepresentationException {
        super(representation);
        this.network = network;
        this.train = train;

        if (network == null) {
            throw new RuntimeException("Neural network should be provided for the fitness evaluation.");
        }

        if (train == null) {
            throw new RuntimeException("Training object should be provided for the fitness evaluation.");
        }

        update();
    }

    /**
     * Constructor used to initialize the chromosome with list of values.
     *
     * @param representation Values as list.
     * @param network        Neural network reference.
     * @param train          Neural network training strategy.
     * @throws InvalidRepresentationException Rise an exception if the values
     *                                        are not valid.
     */
    public WeightsChromosome(List<Double> representation,
                             BasicNetwork network, Train train)
            throws InvalidRepresentationException {
        super(representation);
        this.network = network;
        this.train = train;

        if (network == null) {
            throw new RuntimeException("Neural network should be provided for the fitness evaluation.");
        }

        if (train == null) {
            throw new RuntimeException("Training object should be provided for the fitness evaluation.");
        }

        update();
    }

    /**
     * Constructor used to initialize the chromosome with list of values.
     *
     * @param representation Values as list.
     * @param copy           Deep copy flag.
     * @param network        Neural network reference.
     * @param train          Neural network training strategy.
     */
    public WeightsChromosome(List<Double> representation, boolean copy,
                             BasicNetwork network, Train train) {
        super(representation, copy);
        this.network = network;
        this.train = train;

        if (network == null) {
            throw new RuntimeException("Neural network should be provided for the fitness evaluation.");
        }

        if (train == null) {
            throw new RuntimeException("Training object should be provided for the fitness evaluation.");
        }

        update();
    }

    /**
     * Fitness value update after chromosome representation change.
     */
    private void update() {
        if (network == null) {
            throw new RuntimeException("Neural network should be provided for the fitness evaluation.");
        }

        if (train == null) {
            throw new RuntimeException("Training object should be provided for the fitness evaluation.");
        }

        /*
         * Load weights from the internal representation into the network
         * structure.
         */
        List<Double> weights = getRepresentation();
        for (int layer = 0, index = 0; layer < network.getLayerCount() - 1; layer++) {
            int bias = network.isLayerBiased(layer) ? 1 : 0;
            for (int from = 0; from < network.getLayerNeuronCount(layer) + bias; from++) {
                for (int to = 0; to < network.getLayerNeuronCount(layer + 1); to++, index++) {
                    network.setWeight(layer, from, to, weights.get(index));
                }
            }
        }

        /*
         * Iterate over the training set in order to calculate network error.
         */
        train.iteration();

        /*
         * Total ANN error is used as fitness value. The bigger the fitness, the better the chromosome.
         */
        fitness = -train.getError();
    }

    /**
     * {@inheritDoc}
     */
    @Override
    public double fitness() {
        return fitness;
    }

    /**
     * {@inheritDoc}
     */
    @Override
    protected void checkValidity(List<Double> values)
            throws InvalidRepresentationException {
        if (network == null) {
            //TODO throw new RuntimeException("Neural network should be provided for the fitness evaluation.");
            return;
        }

        /*
         * Length of the values should match the number of weights in the neural
         * network structure.
         */
        int counter = 0;
        for (int layer = 0; layer < network.getLayerCount() - 1; layer++) {
            int bias = network.isLayerBiased(layer) ? 1 : 0;
            for (int from = 0; from < network.getLayerNeuronCount(layer) + bias; from++) {
                for (int to = 0; to < network.getLayerNeuronCount(layer + 1); to++) {
                    counter++;
                }
            }
        }

        if (values == null || counter != values.size()) {
            // TODO Report the size problem.
        }
    }

    /**
     * {@inheritDoc}
     */
    @Override
    public AbstractListChromosome<Double> newFixedLengthChromosome(
            List<Double> values) {
        return new WeightsChromosome(values, true, network, train);
    }

    /**
     * Chromosome representation getter.
     *
     * @return List with chromosome values.
     */
    public List<Double> getRepresentation() {
        return getRepresentation();
    }
}
\end{verbatim}

\textbf{\underline{AnnErrorMinimizationProblem.java}}
\begin{verbatim}
package org.moeaframework.problem.misc;

import org.encog.neural.networks.BasicNetwork;
import org.encog.neural.networks.training.propagation.Propagation;
import org.moeaframework.core.Solution;
import org.moeaframework.core.variable.EncodingUtils;
import org.moeaframework.core.variable.RealVariable;
import org.moeaframework.problem.AbstractProblem;

import java.util.List;

/**
 * ANN error minimization problem.
 */
public class AnnErrorMinimizationProblem extends AbstractProblem {

    /**
     *
     */
    private BasicNetwork network;

    /**
     *
     */
    private Propagation train;

    /**
     *
     */
    private List<Double> initial;

    /**
     * @param solution
     * @param network
     * @param train
     */
    public AnnErrorMinimizationProblem(List<Double> solution, BasicNetwork network, Propagation train) {
        this(solution.size(), 1);
        initial = solution;
        this.network = network;
        this.train = train;
    }

    /**
     * {@inheritDoc}
     */
    private AnnErrorMinimizationProblem(int numberOfVariables, int numberOfObjectives) {
        super(numberOfVariables, numberOfObjectives);
    }

    /**
     * {@inheritDoc}
     */
    private AnnErrorMinimizationProblem(int numberOfVariables, int numberOfObjectives, int numberOfConstraints) {
        super(numberOfVariables, numberOfObjectives, numberOfConstraints);
    }

    /**
     * {@inheritDoc}
     */
    @Override
    public void evaluate(Solution solution) {
        if (network == null) {
            throw new RuntimeException("Neural network should be provided for the fitness evaluation.");
        }

        if (train == null) {
            throw new RuntimeException("Training object should be provided for the fitness evaluation.");
        }

        /*
         * Load weights from the internal representation into the network
         * structure.
         */
        double[] weights = EncodingUtils.getReal(solution);
        for (int layer = 0, index = 0; layer < network.getLayerCount() - 1; layer++) {
            int bias = network.isLayerBiased(layer) ? 1 : 0;
            for (int from = 0; from < network.getLayerNeuronCount(layer) + bias; from++) {
                for (int to = 0; to < network.getLayerNeuronCount(layer + 1); to++, index++) {
                    network.setWeight(layer, from, to, weights[index]);
                }
            }
        }

        /*
         * Iterate over the training set in order to calculate network error.
         */
        train.iteration();

        /*
         * Total ANN error is used as fitness value. The bigger the fitness, the better the chromosome.
         */
        solution.setObjective(0, -train.getError());
    }

    /**
     * {@inheritDoc}
     */
    @Override
    public Solution newSolution() {
        Solution solution = new Solution(initial.size(), 1, 0);
        for (int i = 0; i < initial.size(); i++) {
            solution.setVariable(i, new RealVariable(initial.get(i), -Double.MAX_VALUE + 1, Double.MAX_VALUE - 1));
        }
        return solution;
    }
}
\end{verbatim}

\end{spacing}

% Задна корица.
%\includepdf[pages=-]{covers/back}

\end{document}
